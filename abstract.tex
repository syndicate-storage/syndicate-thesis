The proliferation of commodity cloud services helps developers build
wide-area ``system-of-systems'' applications by harnessing
cloud storage, CDNs, and public datasets as reusable building blocks.
But to do so, developers must contend with two
long-term challenges.  First, whenever developers change storage providers, 
they must work to preserve the application's expected \emph{storage
semantics}, i.e. the rules governing how the application expects the
storage provider to handle its reads and writes.  Today, changing
storage providers is costly, because developers need to patch
the application to make it compatible with the new provider's
data consistency model, access controls, replica placement
strategies, and so on.

At the same time, users have certain expectations about how their
data will be used, which the application must meet.
For example, depending on the application, users
may expect that their data will be kept private from other users, that their data will be exportable
to other applications, that accesses to their data will be logged in an auditable way,
and so on.  In the limit, each user's expectations represent an implicit
\emph{policy} constraining how their data can be stored.  Honoring
these policies is difficult for developers who rely on third-party storage
providers because the storage provider is often unaware of them.

This thesis addresses these challenges with a new wide-area storage paradigm, called ``software-defined
storage'' (SDS), that runs in-between applications and cloud services.
SDS-enabled applications do not host data, but instead let users
bring their preferred cloud services to the application.  By taking a
user-centric approach to hosting data, users are empowered to programmatically
specify their policies independent of their applications and select
services that will honor them.  To support this approach and to
tolerate service provider changes, SDS empowers developers to programmatically
specify their application's storage semantics independent of storage providers.

This thesis presents the design principles for SDS, and validates their
real-world applicability with two SDS implementations and several non-trivial
applications built on top of them.  Most of these applications are used in
production today.  This thesis
presents microbenchmarks of the SDS implementations and uses
real-world experiences to show how to make the most of SDS.
