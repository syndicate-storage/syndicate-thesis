The proliferation of commodity cloud services helps developers build
wide-area ``system-of-systems'' applications by harnessing
cloud storage, CDNs, and public datasets as reusable building blocks.
However, as long as the application is responsible for hosting user data
on third-party services, developers must contend with two
long-term challenges.  First, they must preserve the application's
required storage semantics across an evolving set of service
offerings.  Second, they must do so while respecting their users' data access
policies, such as privacy, portability, and data retention concerns.
These challenges push successful applications to abandon
commodity services altogether in favor of building bespoke services.

This thesis addresses these challenges with a wide-area storage protocol, called ``software-defined
storage'' (SDS), that runs in between applications and cloud services.
SDS-enabled applications do not host data, but instead let users
bring their preferred commodity cloud services to the application.  By taking a
user-centric approach to hosting data, users are empowered to programmatically
specify their policies independent of the applications they use.  SDS 
employs a novel programming technique that expresses storage semantics
as a networked pipeline of reusable 
programs that, when executed on a read or write, recover the application's storage
semantics regardless of the routes taken by the data.

This thesis presents the design principles for SDS, and validates their
real-world applicability with two SDS implementations and several non-trivial
applications built on top of them.  Most of these applications are used in
production settings today, and at least one is revenue-positive.  This thesis
presents early performance results of its SDS implementations and uses
real-world experiences to show developers how to make the most of SDS.
