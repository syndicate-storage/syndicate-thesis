\documentclass[12pt,lot, lof]{puthesis}
\newcommand{\proquestmode}{}
% I prefer proquestmode to be off for electronic copies for normal use, since the colored links are less distracting. However when printed in black and white, the colored links are difficult to read. 

\pdfminorversion=4

\title{Wide-area Software-defined Storage}

\submitted{June 2018} 
\copyrightyear{2018} 
\author{Jude Christopher Nelson}
\adviser{Larry Peterson} 
\department{Computer Science}

%%%%%%%%%%%%%%%%%%%%%%%%%%%%%%%%%%%%%%%%%%%%%%%%%%%%%%%%%%%%%\
%%%% Tweak float placements
% From: http://mintaka.sdsu.edu/GF/bibliog/latex/floats.html "Controlling LaTeX Floats"
% and based on: http://www.tex.ac.uk/cgi-bin/texfaq2html?label=floats
% LaTeX defaults listed at: http://people.cs.uu.nl/piet/floats/node1.html

% Alter some LaTeX defaults for better treatment of figures:
    % See p.105 of "TeX Unbound" for suggested values.
    % See pp. 199-200 of Lamport's "LaTeX" book for details.
    %   General parameters, for ALL pages:
    \renewcommand{\topfraction}{0.85}	% max fraction of floats at top
    \renewcommand{\bottomfraction}{0.6}	% max fraction of floats at bottom
    %   Parameters for TEXT pages (not float pages):
    \setcounter{topnumber}{2}
    \setcounter{bottomnumber}{2}
    \setcounter{totalnumber}{4}     % 2 may work better
    \setcounter{dbltopnumber}{2}    % for 2-column pages
    \renewcommand{\dbltopfraction}{0.66}	% fit big float above 2-col. text
    \renewcommand{\textfraction}{0.15}	% allow minimal text w. figs
    %   Parameters for FLOAT pages (not text pages):
    \renewcommand{\floatpagefraction}{0.66}	% require fuller float pages
	% N.B.: floatpagefraction MUST be less than topfraction !!
    \renewcommand{\dblfloatpagefraction}{0.66}	% require fuller float pages

% The documentclass already sets parameters to make a high penalty for widows and orphans. 

\usepackage[utf8]{inputenc}
\usepackage{subcaption}
\usepackage{graphicx}
\usepackage{url}
\usepackage{verbatim}
\usepackage{multirow}
\usepackage{longtable}
\usepackage{booktabs}

%set parameters for longtable:
% default caption width is 4in for longtable, but wider for normal tables
\setlength{\LTcapwidth}{\textwidth}


%%%%%%%%%%%%%%%%%%%%%%%%%%%%%%%%%%%%%%%%%%%%%%%%%%%%%%%%%%
%%% Printed vs. online formatting
\ifdefined\printmode

% Printed copy
% url package understands urls (with proper line-breaks) without hyperlinking them
\usepackage{url}


\else

\ifdefined\proquestmode
%ProQuest copy -- http://www.princeton.edu/~mudd/thesis/Submissionguide.pdf

% ProQuest requires a double spaced version (set previously). They will take an electronic copy, so we want links in the pdf, but also copies may be printed or made into microfilm in black and white, so we want outlined links instead of colored links.
\usepackage{hyperref}
\hypersetup{bookmarksnumbered}

% copy the already-set title and author to use in the pdf properties
\makeatletter
\hypersetup{pdftitle=\@title,pdfauthor=\@author}
\makeatother

\else
% Online copy

% adds internal linked references, pdf bookmarks, etc

% turn all references and citations into hyperlinks:
%  -- not for printed copies
% -- automatically includes url package
% options:
%   colorlinks makes links by coloring the text instead of putting a rectangle around the text.
\usepackage{hyperref}
\hypersetup{colorlinks,bookmarksnumbered}

% copy the already-set title and author to use in the pdf properties
\makeatletter
\hypersetup{pdftitle=\@title,pdfauthor=\@author}
\makeatother

% make the page number rather than the text be the link for ToC entries
%\hypersetup{linktocpage}
\fi % proquest or online formatting
\fi % printed or online formatting


%%%%%%%%%%%%%%%%%%%%%%%%%%%%%%%%%%%%%%%%%%%%%%%%%%%%%%%%%%%%%\
%%%% Define commands

% Define any custom commands that you want to use.
% For example, highlight notes for future edits to the thesis
%\newcommand{\todo}[1]{\textbf{\emph{TODO:}#1}}


% create an environment that will indent text
% see: http://latex.computersci.org/Reference/ListEnvironments
% 	\raggedright makes them left aligned instead of justified
\newenvironment{indenttext}{
    \begin{list}{}{ \itemsep 0in \itemindent 0in
    \labelsep 0in \labelwidth 0in
    \listparindent 0in
    \topsep 0in \partopsep 0in \parskip 0in \parsep 0in
    \leftmargin 1em \rightmargin 0in
    \raggedright
    }
    \item
  }
  {\end{list}}

% another environment that's an indented list, with no spaces between items -- if we want multiple items/lines. Useful in tables. Use \item inside the environment.
% 	\raggedright makes them left aligned instead of justified
\newenvironment{indentlist}{
    \begin{list}{}{ \itemsep 0in \itemindent 0in
    \labelsep 0in \labelwidth 0in
    \listparindent 0in
    \topsep 0in \partopsep 0in \parskip 0in \parsep 0in
    \leftmargin 1em \rightmargin 0in
    \raggedright
    }

  }
  {\end{list}}



%%%%%%%%%%%%%%%%%%%%%%%%%%%%%%%%%%%%%%%%%%%%%%%%%%%%%%%%%%%%%\
%%%% Front-matter

% For early drafts, you may want to disable some of the frontmatter. Simply change this to "\ifodd 1" to do so.
\ifodd 0
% front-matter disabled while writing chapters
\renewcommand{\maketitlepage}{}
\renewcommand*{\makecopyrightpage}{}
\renewcommand*{\makeabstract}{}

% you can just skip the \acknowledgements and \dedication commands to leave out these sections.

\else


\abstract{
% Abstract can be any length, but should be max 350 words for a Dissertation for ProQuest's print indicies (150 words for a Master's Thesis) or it will be truncated for those uses.
The proliferation of commodity cloud services helps developers build
wide-area ``system-of-systems'' applications by harnessing
cloud storage, CDNs, and public datasets as reusable building blocks.
However, as long as the application is responsible for hosting user data
on third-party services, developers must contend with two
long-term challenges.  First, they must preserve the application's
required storage semantics across an evolving set of service
offerings.  Second, they must do so while respecting their users' data access
policies, such as privacy, portability, and data retention concerns.
These challenges push successful applications to abandon
commodity services altogether in favor of building bespoke services.

This thesis addresses these challenges with a wide-area storage protocol, called ``software-defined
storage'' (SDS), that runs in between applications and cloud services.
SDS-enabled applications do not host data, but instead let users
bring their preferred commodity cloud services to the application.  By taking a
user-centric approach to hosting data, users are empowered to programmatically
specify their policies independent of the applications they use.  SDS 
employs a novel programming technique that expresses storage semantics
as a networked pipeline of reusable 
programs that, when executed on a read or write, recover the application's storage
semantics regardless of the routes taken by the data.

This thesis presents the design principles for SDS, and validates their
real-world applicability with two SDS implementations and several non-trivial
applications built on top of them.  Most of these applications are used in
production settings today, and at least one is revenue-positive.  This thesis
presents early performance results of its SDS implementations and uses
real-world experiences to show developers how to make the most of SDS.

}

\acknowledgements{
I would like to thank my parents who encouraged a lifelong passion for learning,
and my partner Meghan who helped me stay encouraged.
I would like to thank my adviser Larry Peterson and my dissertation
committee for providing much-needed guidance along every step of this journey.
I would also like to thank my research collaborators Illyoung Choi, Jack Williams,
Jack L. Pogue III, Scott Baker, Nirav Merchant, and Wathsala Vithanage for all the efforts they have
put into making Syndicate operational.  Similarly, I would like to thank Muneeb
Ali, Ryan Shea, Aaron Blankstein, Ken Liao, Larry Salibra, and Jack Zampolin for
helping to make Gaia operational.  Last but not least,
I would like to thank my research lab members Andy Bavier, Saphan Bhatia, Mike
Wawwrzoniak, Marc Fiuczynski, and the rest of the PlanetLab team for helping me
navigate graduate school.

The work in this thesis is funded by NSF Award 1541318.

}

\dedication{For Mark, Monica, Luke, and Meghan}

\fi  % disable frontmatter


\begin{document}

\makefrontmatter

\chapter{Introduction}
\label{chap:introduction}

The proliferation of commodity cloud services poses new challenges and
opportunities for hosting data.  On the one hand, the availability of
professionally-maintained services is a boon to developers, since it lets them
offloads the operational burden of hosting application data.  On the other
hand, it is difficult to leverage these services over long
timescales.  Services can appear and disappear, and service operators can
unilaterally change their APIs, pricing, and trustworthiness.
Over long enough timescales, developers will find themselves continuously
patching their applications to accomodate new service behaviors.

This thesis presents a novel storage architecture, called \emph{wide-area
software-defined storage} (SDS), that helps developers
leverage commodity cloud services without this constant churn.
SDS allows developers to specify their
end-to-end \emph{storage semantics} up front, independently of
both applications and underlying services.  The storage semantics define the
rules for processing reads and writes end-to-end, and reside in an architectural
layer ``on top'' of cloud services but ``beneath'' applications.
This thesis presents SDS as an architecture for implementing these semantics, and
shows how developers can realize the benefits of cloud services without the
long-term risks.

\section{The System-of-Systems Approach}

Applications built on commodity cloud services are systems-of-systems.
A \emph{system-of-systems} is a process that that aggregates the
functionality provided by multiple independent networked processes
in order to solve a problem that none of them could
handle on their own.  The most prominent system-of-systems 
is the Internet, which uses peering agreements and the Border Gateway
Protocol~\cite{bgp} to aggregate the routing logic in
multiple autonomous networks to provide a global end-to-end packet delivery
service.

Networked processes that run in the application layer of the Internet can also
be systems-of-systems.  For example, a university Webmail
application is a system-of-systems that 
aggregates DNS, the world's SMTP servers, campus-hosted
Web servers, and a university-wide identity and authentication
system to grant students and faculty access to their email in their Web browsers
(Figure~\ref{fig:chap1-system-of-systems}).  This example illustrates
how application-layer systems can be combined with other
application-layer systems to build new application-layer systems.

\begin{figure}[h]
   \caption{Webmail is a system-of-systems wide-area application.  In order for
   Alice to receive an email from Bob, her university's DNS and SMTP servers
   must coordinate with the global DNS and SMTP networks, and her university's
   identity service and Webmail servers must coordinate to deliver her mail to
   her Web browser.}
   \centering
   \includegraphics[width=0.9\textwidth,page=1]{figures/dissertation-figures}
   \label{fig:chap1-system-of-systems}
\end{figure}

This thesis is concerned with helping developers build user-facing
system-of-systems applications on top of \emph{cloud storage},
\emph{content distribution networks} (CDNs), and \emph{curated data-sets}.
An application would use cloud storage providers to host
its data, CDNs to accelerate data delivery to readers,
and curated datasets to provide better application value.
For example, an application like OpenStreetMap~\cite{openstreetmap}
would host its users' preferred routes, maps, and historic queries in cloud storage,
use a CDN to cache map data in appropriate geographic regions,
and use public weather data aggregated by NOAA~\cite{noaa} to predict how long a commute may take.

In these applications, developers spend non-trivial amounts of implementation effort
on preserving end-to-end storage semantics.  This is because an application's
storage semantics depend on the semantics of each system it uses.  
In order to build a correct implementation, developers must account for
the semantics of their chosen cloud services in the application's design.
For example, Web application servers must coordinate with downstream CDN nodes to ensure that
clients read fresh data.  As another example, scientific computing clusters
must ensure that only PIs can read sensitive data, and only from within the lab.
In today's system-of-systems applications, these semantics may be interwoven
with the business logic.

% TODO: numbers from e.g. Gartner about the growth of the cloud services market?
Despite the difficulty, the growth of the cloud services market and the proliferation of
applications relying on them demonstrates their promise as system-of-systems
building blocks.  Developers do not have to re-invent the functionality they
provide each time they build a new application.  Developers can instead
simply purchase more service capacity to
handle both their existing and new applications' needs.
This reduces time-to-market, speeds up product iteration,
and lowers the barrier to entry for building new applications.

\subsection{Challenges}

The benefits of building system-of-systems with cloud services are overshadowed
by three challenges.  First, \emph{the underlying services
are unreliable in the long-term}.  They can unilaterally
change their pricing, feature-set, APIs, semantics, availability, and
trustworthiness.  Applications that rely on a service can break without warning
when the service changes its behaviors, and in doing so,
cost developers unforeseeable amounts of time and money.

This unreliability is agreed to in the legal terms of service.  The terms of
service for popular services explicitly state that the operators have the ability to affect unilateral
changes.  For example, Dropbox unilaterally broke its API from version 1 to version
2~\cite{dropbox-v2-api-psa}, and Twitter dropped its API only after non-trivial
applications were built to leverage it~\cite{twitter-api-deprecation-psa}.

The second challenge is that \emph{cloud services are heterogeneous}.
Services that fill similar roles do not offer compatible interfaces or semantics.
Without careful planning, the application can become implicitly coupled to the
services it uses by accidentally relying on undocumented or unacknowledged
semantics.  For example, a service designed to use a single Amazon S3 bucket may implicitly
depend on its sequential consistency, and may not be able to simply switch to
using Box.net (which provides eventual
consistency~\cite{consistency-comparison-cloud-storage}).
This creates unexpectedly high service switching costs, making it
difficult for developers to address service unreliability or move to better
offerrings.

The third challenge has to do with the fact that applications span multiple
organizations.  For the purposes of this thesis, an \emph{organization} is a set of computers that
adhere to a single data-hosting policy for the data they produce.  Example organizations
include a user's personal devices, a corporation's workstations, or a lab's scientific
compute cluster.  The data-hosting policy determines the conditions under which
reads and writes are allowed to occur, and addresses concerns such as access
controls, data availability, data durability, replica placement, and so on.
The user or users of an organization set its data-hosting policy.

Successful system-of-systems applications respect each organization's
data-hosting policies.  For example, email allows each
organization with an SMTP server to control the store-and-forward policies
of email messages that pass through it.  As another example, federated applications
like Mastadon~\cite{mastadon}, IRC~\cite{irc}, and XMPP~\cite{xmpp} allow each
organization to set rules on how users' uploaded data gets stored and relayed.
In these cases, the applications preserve each organization's policies
by making organizations responsible for hosting their data.

The challenge posed to developers when building on cloud services is that
\emph{organizations no longer host their data}.  Data hosting is outsourced
to the services, which puts organizations in a difficult position with
regards to enforcing their policies.  Either an organization \emph{completely trusts} the
cloud services in this capacity, or it does not run the
application at all.

Since each organization makes its own choices about trust in cloud services,
it becomes difficult for developers to respect their specific policies.
Developers either have to forgo certain organizations
as customers, or implement complex application logic to accomodate different
policies.

\section{Wide-area Software-defined Storage}

This thesis addresses these challenges by separating storage semantics from both
cloud services and applications.  The rules for processing reads and writes
from application end-points are placed in a common data-exchanging
layer in-between applications and the cloud services.  A system that
implements this layer is called a \emph{wide-area software-defined storage} (SDS) system
(Figure~\ref{fig:chap1-sds-overview}).

\begin{figure}[h]
   \caption{Software-defined storage acts as an intermediate ``narrow waist''
   layer that connects user-facing applications to cloud services.}
   \centering
   \includegraphics[width=0.9\textwidth,page=28]{figures/dissertation-figures}
   \label{fig:chap1-sds-overview}
\end{figure}

A SDS system allows developers
to transparently accomodate heterogeneous cloud services by encapsulating
service-specific interfacing logic inside a ``service driver.''  This
isolates a particular service's behaviors from the rest of the
system and makes its functionality accessible via a common API.  Once the SDS system has a
driver implementation for a service, any SDS-powered application can use it
transparently.

SDS tolerates service failures and preserves organizations' data policies by
allowing developers to control the network paths the data takes from the application to
the services (and vice versa).  Each organization runs its own
service driver instances for storing its data, and developers
route application requests to them by means of an ``aggregation driver.''

The aggregation driver is an SDS-specific programming concept that developers
use to implement end-to-end storage semantics.  Its
programming model borrows from both the UNIX shell programming and software-defined
network programming philosophies.  The developer writes an aggregation driver as a
series of composable ``stages,'' which are evaluated in sequential order by the
SDS system to process an application request according to the desired semantics.
Each organization runs one or more stage instances in order to ensure
that its users' reads and writes are processed according to its data-hosting
policy.

The resulting system is one that preserves end-to-end storage semantics,
preserves per-organization data policies, and tolerates service failures.
The SDS system preserves the end-to-end storage semantics by ensuring that
all reads and writes pass through the correct sequence of aggregation driver
stages.  At the same time, the system
lets developers control which organizations' aggregation stage and service driver
instances are utilized to process a given request.  This preserves each
organization's ability to enforce its data policies without being required to
host the data themselves.  Handling service failures is achieved as a
consequence of exposing request-routing decisions to the aggregation driver:
each stage can ``route around'' service or network
failures by choosing a different next-hop stage or service driver instance.

\section{Contributions}

The architecture put forth in this thesis is informed by two real-world SDS
implementations and three sample applications.  The implementations were 
designed to accomodate two sets of real-world use-cases.
The design principles described in this thesis 
were formulated only after the implementations were tested and
deployed in production settings.  The thesis claims the following contributions:

\begin{itemize}

\item This thesis presents the design principles of software-defined storage, framed in
terms of prior work and the real-world storage needs of existing applications.
Adhering to these design principles reduces the man-hours required to keep applications compatible
with existing services while both preserving end-to-end storage semantics and
respecting each organization's data-hosting policies (Chatper~\ref{chap:design_principles}).

\item This thesis presents the design and implementation of two SDS systems: Gaia and
Syndicate.  Syndicate is a real SDS system being deployed in scientific
workflows today, and Gaia is a real SDS system being deployed to build
``serverless'' Web applications (i.e. Web applications that can operate
without the need for application-specific servers).
This thesis shows how Gaia and Syndicate make use of SDS design principles
(Chapter~\ref{chap:syndicate_sds}).

\item This thesis shows how to build SDS-powered applications.  The design and
implementation of non-trivial SDS-powered applications \emph{that could not
have been feasibly built without SDS} are presented.  Among these are an end-to-end encrypted
Webmail client that removes the user from key management, a server-less
groupware application that lets users control how their data gets hosted and
accessed, and a scientific data-staging application that
automatically makes fresh datasets available from existing data repositories to
HPC clusters via commodity CDNs.
(Chapter~\ref{chap:applications}).

\item This thesis presents early performance numbers for Gaia and Syndicate, both in the
form of microbenchmarks and in real-world performance of applications built on
top of them (Chapter~\ref{chap:evaluation}).

\end{itemize}


\chapter{Design Principles of Software-defined Storage}
\label{chap:design_principles}

We are used to thinking about storage architecture in terms of layers.
The higher layers implement more specialized interfaces tailored to the needs of
applications (e.g. filesystems, tables, key/value pairs),
while the lower layers offer simpler and more general-purpose interfaces (e.g.
blocks, packets).

This layered perspective is ill-suited for reasoning about
software-defined storage for two reasons.  First, because
the control-plane logic \emph{wraps} the
data-plane logic, control-plane and data-plane communication defies
a layered representation.  Second,
layers cannot represent the relationships between principals and execution
contexts, since different principals execute different aspects of the
control-plane logic.  As an alternative, we present an architectural
representation called the \emph{flow-oriented approach}.

\section{The Flow-oriented Approach}

The flow-oriented approach is concerned with reasoning about storage in terms of
how data passes through the system.  We introduce the concepts of
\emph{administrative domains}, \emph{trust domains}, and \emph{gateways}
to capture the relationship between the
end-to-end paths a datum's bytes take and the principals and computers that
process them along the way.

Using these concepts, we give the data-plane owner
a way to reason about \emph{global} control-plane logic in terms of a set
of \emph{local} interaction functions and execution contexts.
This is a property we take for granted
in centralized and user-centric applications today, since the control-plane
logic is already global in scope and runs only on the data owner's computers.

\subsection{Control-plane Program Control Transfer}

Each principal who interacts with a datum (including the owner) has at least one
interaction function and execution context assigned to them by the owner.  Because
they work together in SDS to implement a global control-plane logic, their
runtime behavior must be consistent with it.  In practice,
this means that interaction functions may need to transfer
program control to other interaction functions, such as through RPC or
message-passing.

There are two key reasons why this will be commonplace.  First, in order to keep data
consistent, the control-plane logic must \emph{constrain the execution order
of operations} so as to preserve the desired consistency.  This will require
coordination across a subset of interaction functions, which means
transferring program control flow between them.

For example, enforcing linearizability requires the control-plane logic to
execute each data operation atomically with respect to all other operations.
If the control-plane logic has multiple instances of execution---be
they threads, processes, or multiple servers---then the instances must coordinate to
to preserve this invariant (e.g. through locking, message-passing).  This is true regardless
of whether or not the control-plane logic all resides on the owner's computers
(as in traditional user-centric storage) or is distributed across multiple
execution contexts (as with control-plane offloading).

The second reason is that in order to properly enforce access to data, the control-plane logic
must constrain which principals can execute which \emph{aspects} of each
operation.  Since an operation can have multiple aspects, one interaction function
may need to transfer control to another to implement it correctly.

For example, handling a application-level data operation like ``read'' 
may entail carrying out tasks like logging access to private servers or billing one of
the user's bank accounts.  Executing the code for these aspects cannot reasonably be entrusted to the
user.  With traditional user-centric storage, this is
straightforward:  only the data owner's servers and their delegates may handle
all operations.  With control-plane offloading, the data owner will need to
consider what each principal can do with their interaction functions, and
distribute them accordingly to implement the desired operation safely.

TODO: figure of example

\subsection{Administrative and Trust Domains}

Reasoning about program control transfer
requires thinking about the overall system's
administrative and trust domains.  An \emph{administrative domain} is the set of
all computers under the control of one principal, as dictated by the interaction function's
$user$ parameter.  A \emph{trust domain} is
the set of computers that may execute a particular operation on a particular
datum (dictated by the $operation$ and $datum$ parameters).

Administrative and trust domains help the data owner reason about the OS processes that
run interaction functions in their execution contexts.  This is crucial, since
these SDS-managed processes serve as the
the ``access points'' to data for other local processes like application
clients.

In order to ensure correct program control transfers, the SDS system
needs a mapping from the
data owner that specifies where it is safe to place interaction functions.
To do so, we introduce the concept of a \emph{gateway}.

\subsection{Gateways}

A gateway is an aggregate schedulable entity that runs a particular user's
interaction functions.  A gateway belongs to exactly one administrative domain
and exactly one trust domain.  A user may have multiple gateways.

A gateway has two responsibilities.  First, it manages the OS resources
necessary to instantiate and run interaction functions in their proper execution
contexts.  Second, it coordinates with other gateways in the same trust domain 
to ensure that each data interaction is processed by the data owner's
most-recently-specified interaction functions and execution contexts.

TODO: explain how we ensure that *only* the data owner's functions can run.  May
need to delve into the certificate graph concept here.

\subsection{Data Flows}

TODO: you can build a ``pipeline'' of gateways that let you construct the global
control-plane logic.  Gateways coordinate to ensure that each interaction is
done safely--that is, an interaction can only take place of each gateway that
processes it has the most recent interaction function and execution context from
the data owner (i.e. the distributed control-plane logic is consistent).

\subsection{Discussion: Why Bottom-up is Better than Top-down}

It is tempting to consider a ``top-down'' approach to implementing a distributed
control plane by first writing a global control-plane program and compiling it into
distributed interaction functions.  While this
approach is possible, our experience with deploying and running SDS 
has led us to believe that a ``bottom-up'' approach is superior.  We argue that
the programmer should implement the control-plane logic by writing the
interaction functions that, when executed together, yield the desired global
control-plane behavior.

Our reasoning has to do with the presence of administrative and trust domains in
the storage system model.  We have found that administrative domains are often
rigid, since a principal's set of computers does not change often.  However,
trust domains are often fluid.

There are two big sources of this fluidity.  First, users are frequently granted
and later denied access to data on a ``need-to-know'' basis.  For example, this 
is true for VMs in a science cluster, each of which constitutes a user.  Second, a data
owner regularly modifies her control-plane logic to fix bugs and add features,
and as such regularly pushes new code to users that alters how they process data
interactions (including the paths that data may take).

The bottom-up approach is more amenable to providing a fluid environment.
By treating cross-function control-flow transfers as something the
data owner must directly invoke in her code, we give her as much leeway as possible
in deciding when and how her users' computers communicate.  This is proven
helpful in designing for performance-sensitive workloads, where limiting network
requests is crucial.  It has also proven helpful in debugging control-plane
behavior on a data path, since it is obvious from the interaction function
implementations as to which computers execute which code.  In addition,
executing a control-transfer has proven to be easy to abstract away with a shared library or a
subprocess helper program, which makes it easy for data owners to use existing
languages, libraries, and tools to create interaction functions.

By contrast, a top-down approach would constrain the data owner 
to using a domain-specific language, compiler, and
debugger.  The language would differ from existing languages because it would
need to be sufficiently expressive to define which code branches execute in which
administrative and trust domains, given the current operation, user, and datum.
This is an equally powerful approach, but it is one that we did not explore due
to practical constraints.

\section{Strategies for Gateway Coordination}

TODO: talk about the metadata service, and alternatives to it

\section{Programming Model}

\subsection{North-bound Interface}

TODO: this is the filesystem, the database, etc.

\subsection{South-bound Interface}

TODO: this is the storage driver programming model.  It's chunk-oriented, with
immutable chunks

\section{Scalability}


% \subsection{Gateways and Data Flows}

% First, interaction functions are not only distributed across an unreliable,
% untrusted network, but also distributed across multiple \emph{administrative
% domains}.  

% Second, the global control-plane program implicitly branches to code running
% within the untrusted storage infrastructure.

% we take a bottom-up approach: do not start with a global program and try to
% programmatically break it apart into interaction functions (that's going to be
% one hell of a leaky abstraction).  instead, take a bottom-up approach: program
% interaction functions, and make them send continuations to one another to
% transfer flow control across the network and/or between users (required in order
% to reason about infrastructure)
% 
% 
% 
% These pre-defined interaction functions implement the
% functionality offered by the infrastructure.  For example, a cloud storage
% provider has a pre-defined interaction function that stores a persistent copy of
% all bytes it writes.  As another example, a CDN has a pre-defined interaction
% function that caches a copy of bytes read and evicts them at a later time.
% Even though their implementations are set by the infrastructure operators, the
% fact that the data owner explicitly allows them to operate on her data is
% equivalent to her generating the functions and giving them out to them.
% 
% % this is wrong.  Control-plane functions "compose" in continuation-passing
% % style.  One data flow represents one possible trace of a "global control
% % program" that is physically distributed.  There exists an edge between IF 1
% % and IF 2 if IF 2 can "pick up" where IF 1 leaves off--i.e. the end of IF 1 can
% % branch to IF 2.  Show that the set of IFs is analogous to possible if/then
% % branches within a global control program (show IFs within a global control
% % program).
% The interaction functions themselves are derived from a global control-plane
% program, which we assume is expressed in terms of how to handle a specific
% operation by a specific user on a specific datum.  This does not imply that each
% interaction function may execute in parallel, however, since in the global
% program there may exist data dependencies between them.  As such, when
% generating interaction functions, the data owner may design them 
% 
% We call a path from the data owner's interaction function to a user's interaction
% function a \emph{data flow}.  
% 
% Data flows and interaction functions are the fundamental building blocks of
% software-defined storage.  The functional composition of the sequence of
% interaction functions along a data flow fully describes the execution of the
% control-plane logic that operates on the datum's bytes (including the logic for 
% determining authoritative, consistent replica state).  At the same time, the
% association of each interaction function with particular principal captures the
% notion of access control:  a principal---be it a user or a storage
% infrastructure operator---only interacts with a replica
% in exactly the manner prescribed by its associated interaction function.
% 
% We call this strategy the \emph{flow-oriented approach} to storage
% design.  The advantage of the flow-oriented approach is that it cleanly
% represents data ownership:  a user owns a piece of data if and only if all of
% its data flows are set by her and only her.
% 
% \section{Interacting with Data}
% 
% An interaction function may either implement a ``read'' or a ``write''
% operation.  The ``read'' operation receives a sequence of bytes from the
% data-plane and translates them as a consistent, authoritative replica.
% The ``write'' operation generates a sequence of bytes such that it can be later translated
% into a replica by a ``read,'' and then sends them on the data-plane.
% 
% The bytes themselves are sent and received by the data-plane logic, which is implemented as a
% storage driver that the interaction function wraps.  To express this
% relationship, we first define the storage
% driver's data-plane logic in terms of ``get'' and ``put'' operations, and then
% define the interaction functions as higher-order functions on them.
% 
% \subsection{The Data-plane Model}
% 
% In the flow-oriented approach, the data-plane is a logical read/wrote
% medium.  The purpose of a storage driver is to implement an
% associative array interface on top of it.  Its two operations, $GET$ and $PUT$, are
% defined as follows:
% 
% $$GET(datum_id: str) -> bytes or error$$
% \\
% $$PUT(datum_id: str, data: bytes) -> bool$$
% 
% $GET$ takes a datum ID as its argument and returns the associated sequence
% of bytes for the identified data (or an error message on failure).
% $PUT$ takes a datum ID and a byte string
% as arguments and returns a boolean value to indicate whether or not the bytes
% successfully written to the data-plane.  $PUT$ is idempotent.
% 
% The storage driver must implement $GET$ and $PUT$ to offer a
% consistency model that is compatible with per-key read-follows-write consistency.
% That is, a $PUT$ on a particular datum ID followed by a $GET$ on the same datum
% ID must result in the $GET$ returning the bytes stored by the $PUT$.  This
% invariant must hold regardless of which clients execute them.
% 
% This is a reasonable requirement in practice.  Commonly-used cloud storage
% infrastructure~\cite{s3}~\cite{google-drive}~\cite{onedrive} claim to offer this
% consistency model already, and implement $GET$ and $PUT$ operations as
% part of their client software offerrings.
% 
% We are not concerned about how the storage driver implementation or
% the infrastructure handle conflicting $PUT$ operations.  This is because
% in our construction of the control-plane logic, we will guarantee that a datum
% will be $PUT$ at most once (see the next section).  We will use this property
% to allow the data owner to implement arbitrary consistency models regardless
% of the infrastructures' abilities to do so.
% 
% \subsection{The Control-plane Model}
% 
% The control-plane logic is constructed from both ``read'' and ``write''
% interaction functions.  They are higher-order functions that take a particular
% $GET$ or $PUT$ function as arguments.
% 
% Control-plane functions do not operate on raw bytes, but instead operate on
% replicas.  A replica is an abstract composite data type that combines an
% application-specific payload type with consistency metadata and
% a proof of authenticity.  The consistency metadata may include things like a
% Lamport clock, a vector clock, an operation history, and so on.  The proof of
% authenticity can include 
% 
% We define them as follows:
% 
% $$READ(user_id: str, datum_id: str, get_func: GET) -> replica: Replica$$
% \\
% $$WRITE(user_id: str, datum_id: str, data: bytes) -> bytes$$
% 
% \subsection{Example}
% 
% Consider the simple SDS system in
% Figure~\ref{fig:design-flow-oriented-architecture}.  The data owner designs and
% implements interaction functions 1, 2, and 3, and gives them respectively to
% users 1, 2, and 3.  At the same time, the data owner selects a commodity cloud
% storage provider to host his data, and a commodity CDN to cache data on behalf of
% other users.
% 
% The owner's ``write'' interaction function is designed to send the bytes he writes as
% input to the storage provider's interaction function.  The storage provider's
% interaction function replicates to persistent media.  When user 1 later attempts
% to read, his ``read'' interaction function forwards the request to the storage
% provider's ``read'' function, which replies the bytes.  User 1's ``read''
% function interprets these bytes into the requested replica.
% 
% \begin{figure}[ht!]
%    \centering
%    \includegraphics[width=0.9\textwidth]{figures/design-flow-oriented-architecture}
%    \caption{\it Overview of flow-oriented architecture.  The ``storage'' and
% ``cache'' interaction functions are implemented by the infrastructure operators,
% but nevertheless incorporated into data flows by the owner by virtue of the fact
% that the owner has designed the other interaction functions to compose with them.}
%    \label{fig:design-flow-oriented-architecture}
% \end{figure}
% 
% When either user 2 or 3 attempt to read, their ``read'' interaction functions forward
% their requests to the CDN's ``read'' interaction function.  The CDN in turn
% forwards the request to the storage infrastructure's ``read'' function.  When
% the storage provider replies the bytes, the CDN's function caches them before
% forwarding them back to the users' ``read'' functions.
% 
% 
% 
% % TODO: table of interaction functions in this diagram
% 
% % client organization
% % gateway: the entity that loads and executes interaction functions and storage drivers
% % metadata service: a way to transfer flow control across interactionn functions
% 
% % data organization
% % data writes --> reassemble into consistent views by means of a manifest
% % read: get latest manifest, get writes that went into it
% % write: make new write runs, get owner to incorporate them into a new
% % manifest

\chapter{Software-defined Storage Systems: Design and Implementation}
\label{chap:syndicate_sds}

In order to vet our design principles for software-defined storage, we used them
to design and implement two production-quality systems.  The first system,
called Gaia, implements a global key/value store with programmable semantics for its
``get'' and ``put'' operations.  The second system, called Syndicate, implements
a full POSIX filesystem interface with programmable semantics for most
filesystem operations.

\section{Gaia: a Global Key/Value SDS System}

Gaia is a global key/value store designed for allowing users to host their data
for decentralized applications.  We say ``decentralized applications'' to mean
applications where all of the business logic computations occur on the
users' computers.  For example, a decentralized todo-list
application~\cite{blockstack-todo} would fetch the user's application state from
the storage providers of their choice, allow the user to interact with the items
once loaded, and would store the resulting state back to the storage providers
when the user is done.  Unlike competing applications like Google
Calendar~\cite{google-calendar}, there is no ``application server'' that
runs business logic computations over the user's data.

Gaia is designed primarily for Web programming environments, and offers two
modes of operation.  The first mode, called ``single-player mode,'' offers
behavior similar to what Web developers today expect from HTML5
\texttt{localStorage}~\cite{w3c-localstorage}.  The Web code can load a value
given a key and store a (key, value) pair, with the expectation that only this
instance of the code will be able to interact with the key.  This is the mode
used by our example todo-list application---a user only interacts with their own
todo-list data, and users cannot read or write to each other's data.

The second mode, called ``multi-player mode,'' offers one-writer many-readers
semantics.  Only a user may write to their own keys, but any user may discover
and read their keys.  For example, a decentralized blogging application would
use Gaia's multi-player mode to allow a user to publish blog posts, and allow
other users to read them.

The main contribution of Gaia is that it gives Web developers a secure and
reliable way to outsource data-hosting to users.  Gaia ensures that each user
securely discovers each other users' \emph{current public keys} and \emph{data
source URLs} for each application they use prior to loading the data.
In doing so, Gaia offers end-to-end
data authenticity and confidentiality while using untrusted commodity cloud infrastructure
to host and serve application data.

We present Gaia as a proof-of-concept system for demonstrating SDS design
principles.  It is a production system today and is used by
Blocksatck~\cite{applications} as the primary way for hosting user data.

\subsection{Motivation}

In designing Gaia, we set out to ensure that users ``own'' the data that they
write.  By this, we mean that a user unilaterally decides where authoritative
copies of their data are hosted, and how can access them.  In doing so, we 
(1) allow users to keep their data in the event that the application disappears, we
(2) allow users to shop around for different competing applications by granting them
access to their data, and we (3) allow developers to avoid the operational burden of
hosting any state on their users' behalf.

In many cases, a Web application's data interaction model is centered around
individual user activity.  Users can read and write their own data, but the can
at best read other users' data.  Users rarely have the ability to write to the
same data, and each datum belongs to at most one user.

For example, this is
true of the data interaction models for most social media applications,
most photo-sharing applications, and most blogging applications.  In systems
that present ``shared-write'' views of data, like a comment section on a blog or
a shared Google Document page~\cite{google-docs}, the data interaction model
nevertheless attributes each write to a specific user (and then ``merges''
writes to present a consistent view).  In all these cases, each write is
attributed to exactly one user.

This model for data storage is
inspired by the model seen in desktop application storage today:  the user holds all
desktop applications' state on their hard drive, instead of replicating it to
the application developers' remote servers.  Gaia takes the additional step of
allowing users to read each other's data if the writer permits it.

Even though a Web application has a single logical database that holds all of
its users state, our observations about its data interaction model allows us to reformulate
the global database as a collection of single-writer multi-reader user-specific
databases.  It becomes the application client's job to translate a set of reads across
the users' databases into a consistent view (whereas this had traditionally been
done by the application's global database).

This reformulation of application storage gives us the ability to decouple each
users' database from the administrative domain in which it runs.  The
application only needs to be able to read from a user's database in order to
present other users with a view of its data.  The database need not reside on
servers that the application developer chooses.

Our SDS design principles come into play in the following tasks:
\begin{itemize}
   \item \textbf{Multiple Storage Systems}.  We will show how Gaia allows users
      to choose the storage systems that will host their data in an
      application-agnostic way.
   \item \textbf{User Storage Policies}.  We will show how Gaia allows users to
      stipulate programmatic policies pertaining to data availability and durability.
   \item \textbf{Application-specific Views}.  We will use SDS aggregation
      drivers to show how to construct a
      global, consistent view of a set of users' databases.
\end{itemize}

\subsection{Blocks, Manifests, and Volumes}

Gaia organizes a user's data into a set of volumes, where each volume
holds one application's data.  The user decides whether or not the volume
operates in single-player or multi-player mode, and selects the set of
service drivers to use to replicate data.

A volume in Gaia is a key/value store.  The API the Gaia client exposes to
programmers resembles HTML5's \texttt{localStorage}.  It offers three methods:
\texttt{get(key) -> value}, \texttt{put(key, value) -> bool}, and
\texttt{delete(key) -> bool}.

Internally, the set of keys in a volume are bundled into a single manifest.
Each value is the associated block.  This means that writes are serialized
across keys, and that writes to a key are atomic.  These behaviors were chosen
specifically to emulate the \texttt{localStorage} API contract.

The service drivers for a volume must implement at least per-key sequential
consistency.

\subsection{Gateways}

% TODO: figure of Gaia

Whenever the application client writes new data, the new block and manifest are
generated within the Web browser and sent to a co-located Gaia node
(Figure~\fig{fig:gaia-overview}).  The Gaia node implements an SDS gateway for
each application volume.  Upon receipt of the new data, it loads and
synchronously invokes the volume-specific service drivers to replicate the key
and value to the users' back-end storage providers.  When the application later
reads the key, the user's Web browser contacts the Gaia node to (1) load up its
volume, and (2) load and return the value for the requested key.

Users run a Gaia node for each of their devices.
In Gaia, we make the consistency assumption that \textbf{at most one device will
be writing to a given volume at any given time}.  This is a reasonable
assumption in practice, because (1) a volume may only be written to by the user
that owns it, and (2) a user typically does not access the same application from
two different devices simultaneously.

We use this assumption to side-step the need for a user's Gaia nodes to
coordinate to resolve write-conflicts and coordinator-changes.  Since writes are
always serialized through the user, they will not conflict.  The writer Gaia
node is always the coordinator, and the current coordinator for a volume in Gaia
is simply the last writer.

\subsection{Metadata Service}

A user's Gaia nodes
implement a peer-to-peer metadata service.  The Gaia MS is based on prior work
on Blockstack~\cite{blockstack}~\cite{virtualchain}~\cite{ali2017}.  It enables
Gaia nodes to both ensure that readers do not see stale data, and to ensure that
any Gaia node can discover and read keys from a given volume for any user.

Gaia uses a blockchain-based SSI system both for bootstrapping trust between
users and for implementing the ``volume discovery'' functions of its MS.  When the
user registers a name in the SSI, she includes a cryptographic hash within the
blockchain record.  This hash corresponds to a ``zone file'' that
contains URLs to the user's list of Gaia volumes.

Gaia nodes work with the SSI system to find both find the set of names and public keys
as well as the set of zone file hashes.  They self-organize into an unstructured
peer-to-peer network called the Atlas network~\cite{blockstack-white-paper}
through which they exchange zone files.  They exchange
bitvectors with one another to announce the availability of their zone files,
and exchange zone files with one another in rarest-first order such that all
Gaia nodes eventually have a 100\% replica of all zone files.

Since Gaia nodes view the same blockchain, they calculate the same sequence of zone
file hashes.  This gives them a ``zone file whitelist'' that grows at a constant
rate (no faster than the blockchain).  They use the whitelist to identify only
legitimate zone files, and rely on the blockchain to ensure that not too many
new zone files can be introduced into the system at once.  A detailed
description of the peer network can be found in~\cite{ali2017}.

% TODO: figure of Gaia volume lookups

The Atlas network ensures that each Gaia node both knows the current public key
and current zone file for each user.  Each user's
zone file points to a set of signed JSON web tokens.  Each JWT contains the
public keys for each of the user's devices, the public keys and URLs to
replicas of each of the user's volume descriptions.  This way, a Gaia node can
look up an application-specific volume for a user given the user's name on the
SSI system~\ref{fig:gaia-volume-lookups}.  Importantly, the networks and storage
providers hosting zone files, JWTs, and volume descriptions are \emph{not} part
of the trusted computing base.

\subsection{Aggregation Drivers}

Users specify end-to-end storage semantics in Gaia by standing up and running
publicly-routable Gaia nodes to process their writes, and handle reads from
other users.  To do so, the user registers a name for the Gaia node in the SSI
system and records an IP address for it in the name's associated zone file.
Then, when the user creates a volume, she simply lists the Gaia node's SSI name
in the volume description as the "read" endpoint.  When other users go to
read from her volume, their Gaia nodes issue the request to the user's Gaia node
indicated in the volume description.  This allows the user to control all access
flows to her data.

Gaia nodes process data flows by inspecting each other's zone files.
Each node's zone file contains a ``next hop'' node to contact for access and
mutation flows, as well as a URL and hash of the code that it will execute to process
the chunks.  This gives an application a global view of what will happen to its
chunks when it reads or writes data.

% TODO: read and write diagram

When handling an access flow, the user's Gaia node first inspects the volume
record to determine the next-hop.  It
forwards the request to this Gaia node (looking up its IP address in the SSI
system), which in turn may either service the request for data, or forward it
along as well.  This resolution process continues recursively, until a Gaia node
loads data from a storage provider.  When it returns the chunks, the Gaia nodes
along the request path execute their application-specific stages to process it
en route back to the reader~\ref{fig:gaia-reads-and-writes}.

Writes work in a similar fashion.  The user's volume record identifies the
``write'' Gaia node to which to forward new chunks.  Upon receipt of chunks to
store, the Gaia node executes its mutate flow stage logic and forwards the
resulting chunks on to a ``next-hop'' Gaia node.  Eventually, the chunks reach a Gaia
node that will replicate them to underlying storage systems.

The code for each stage is identified by its cryptographic hash in the node's
zone file.  This allows the application to inspect the path of Gaia nodes that
will process access and mutation flows, and determine that the set of nodes are
correctly configured before executing the read or write.

Each Gaia node operator can reprogram the node's behavior by updating the zone file.
This ensures that all other nodes will see the change to the node's behavior (or
at least see that the code may have changed, if only the zone file hash is
discovered by the SSI system's blockchain indexing).  Upon noticing that their
zone file has changed, the Gaia node fetches and installs the new code from a
well-known URL resource record listed in the zone file contents.

\section{Administration}

Administrating Gaia volumes is designed to be straightforward.  Our work on
administration revolves around \emph{removing} unnecessary control points from 
both the user and the developers.
Specifically, the only administrative contact a user has with their volumes is
in connecting storage providers (which is handled via a provider-specific Web UI
that automates creating and sharing an OAuth2~\cite{oauth2} token).

Application developers do not interface directly with storage providers, but
instead with the user's designiated Gaia node (usually a locally-hosted daemon,
but optionally a cloud-hosted daemon if the device, such as a smartphone, cannot
run daemons).  Instead, developers specify the storage requirements the
application needs, and the Gaia node pairs the requirements with storage drivers
when creating its volume.  A table of storage requirements can be found in
Table~\ref{tab:gaia-storage-requirements}.

% TODO: table of Gaia storage classes

Application developers discover a user's Gaia node as part of the sign-in
process.  The sign-in service identifies to the application the network address
of the user's Gaia node.  The application then learns the set of Gaia storage
providers, and the set of capabilities they offer (which can be matched to
storage requirements).

The resulting storage administration workflow for users and developers works as
follows:
\begin{itemize}
   \item When the user creates an account in the SSO service, she connects one
      or more storage providers to her account.
   \item The user loads the application and clicks its "sign-in" UI element.
   \item The application redirects the user to the SSO service's "sign-in" UI,
      which prompts the user to authorize the sign-in request.  Specifically,
      the user is presented with the application's request for either a
      ``single-player'' or ``multi-player'' volume.
   \item Once approved, the SSO service redirects the user back to the
      application, passing it a session token which identifies the user's Gaia
      node.
   \item The application requests a volume.  If this is the first such request,
      the Gaia node creates an application-specific volume.  The node then
      returns a handle to the volume which the application subsequently uses to
      load, store, and delete keys.
\end{itemize}

At no point are users asked to interact with cryptographic keys, and at no point
are users asked to perform access controls.  At no point are the developers
asked to identify or bootstrap a connection to storage providers, and at no
point are developers required to perform any access controls beyond setting up
``single-player'' versus ``multi-player'' storage.

\section{Syndicate: A Scalable Software-defined Storage System}

Syndicate is a scalable software-defined storage system meant for scientific
workloads.  Unlike Gaia, Syndicate is designed to provide shared volumes that
efficiently leverage CDNs for read loads and 
support I/O from a scalable number of concurrent users.  This makes
it ideal for sharing data across compute clusters, where the data sources and
sinks reside in different organizations.

\subsection{Motivation}

Science research is increasingly data-driven and increasingly distributed.
Researchers often share large datasets with other labs across the world and 
with the public.  As the cost of storage space becomes cheaper, scientists can
afford to generate and retain larger and larger amounts of data for the
indefinite future.

These trends create an interesting set of operation challenges:

\begin{itemize}
   \item How do scientists onboard new users and labs that use different
      technology stacks than their own?
   \item How do scientists keep legacy data-processing workflows running in the face
      of changing storage and compute systems?
   \item How do scientists take advantage of commodity storage and
      compute technologies without having to write a lot of bespoke code
      to do so?
   \item How do scientists enfoce data access and retention policies when the
      underlying storage substrate can be changed out from under them?
\end{itemize}

The standard practice today is messy.  Each time a lab wants to change its
storage system, it must re-work its workflows to be compatible.  This entails
more than patching the code to read and write data.  It also means changing their
operational practices for staging data for computations and changing the way they
share data internally and with other labs.

The recent ``containerized approach'' to using relocatable containers, VMs, and
SDNs to preserve the runtime environment for scientific workflows
is a step in the right direction.  However, this only addresses preserving
``localhost'' runtime compatibility.  It does not address sharing data across
compute instances.

Syndicate is designed to fill in this gap.

\subsection{Gateway Types}

Syndicate offers multiple types of gateway.  This is because
different hosts and organizations sharing scientific data have different
responsibilities.  However, these responsibilities can be bucketted into 
at least three common categories:  data acquisition, replication, and user interaction.
As such, Syndicate's default deployment comes with three types of gateway:

\textbf{Acquisition gateways} (AGs) are gateways that connect to an externally-hosted
dataset and ``import'' its records into a Syndicate volume in a read-only
fashion.  It does so by crawling its backend dataset, and publishing metadata
for each (logical) record to the Syndicate MS.  Other gateways read the dataset
by first discovering the metadata, and then asking the AG for the manifest and
chunks (which it generates on-the-fly by fetching data from its backend
dataset).

The interconnection between the dataset service and the gateway is
handled by the AG's service driver.  For example, we have implemented service
drivers for iRODS~\cite{irods}, SQL databases, and FTP-hosted datasets like
GenBank~\cite{genbank} and M-Lab~\cite{mlab}.

AGs offer a file-oriented view of the datasets.  Each record appears as a
logical byte array and are grouped into a hierarchy of directories.

\textbf{Replica gateways} (RGs) are gateways that connect to existing storage
systems.  They provide a read/write interface at the chunk granularity.  We have
implemented service drivers for Amazon S3~\cite{s3}, Dropbox~\cite{dropbox},
Google Drive~\cite{google-drive}, Amazon Glacier~\cite{amazon-glacier}, iRODS,
and local disk (for compatibility with NFS~\cite{nfs}, AFS~\cite{afs},
Ceph~\cite{ceph}, and other legacy distributed filesystems used today).

\textbf{User gateways} (UGs) are gateways that connect users and their workflows
to other gateways.  Each UG provides a different interface to workflows, subject
to their needs.  For example, we have implemented a UG that implements a
FUSE~\cite{fuse} filesystem, a UG that implements a RESTful~\cite{rest}
interface, a UG that implements a suite of UNIX-like shell utilities, and a UG
that implements a Hadoop filesystem~\cite{hadoop} backend.

Syndicate allows operators to specify new gateway types at runtime, allowing
them to incrementally deploy and adapt the system to changing workloads.  Each
gateway's type is embedded in their certificate, so each gateway knows at all
times the network addresses and types of all other gateways in the volume.
This is useful for both scaling up the number of requests a gateway can handle,
and for creating distributed implementations of aggregation driver stages.  We
explore examples in Chapter~\ref{chap:applications}.

\subsection{Data Organization}

Unlike Gaia, each record in a Syndicate volume has its own manifest, and is
comprised of a variable number of blocks.  The block size is fixed for the
volume, but each volume can have its own block size.

It is important to distinguish between the \emph{logical} representation of a record,
the \emph{application} representation of the record, and the 
\emph{on-the-wire} representation of the record.  The logical representation is
the view of the data within a gateway (i.e. the ``narrow waist'' that connects
the application representation to the on-the-wire representation).  In this
representation, each record appears as a flat byte array (i.e. a file), with
fixed-sized blocks and one manifest.

Application-facing gateways (i.e. UGs in Syndicate) are free to represent data 
to the application in any way they want.  For example, an UG implementation
may represent a data record as a
SQL database.  Such a UG would require applications to interact with the data
via SQL commands.  The implementation would translate the commands into
\texttt{get}, \texttt{put}, and \texttt{delete} operations over the record's
blocks at the logical layer.

In addition, Syndicate's aggregation driver model gives gateways the ability to
control a record's chunks' on-the-wire representation.  % TODO

Volumes in Syndicate can have arbitrarily many data records, and each data
record may have arbitrary sizes (i.e. made of arbitrarily many blocks).
Manifests, blocks, and certificates are all cacheable for
indefinite amounts of time, since Syndicate ensures that they are all immutable
(that is, they each receive new IDs in the system when their contents change).

Readers construct URLs to manifests, blocks, and certificates using their IDs to
ensure that any intermediate caches serve the right data.  Readers learn the IDs
directly from the MS, and use in-band hints to determine when their view of
these IDs is stale (as described in Chapter~\ref{chap:design_principles}).

\subsection{Data Flows}

Syndicate gateways route requests to one another based in part on what their
type is.  Specifically,
UGs initiate access flows to AGs and RGs to handle reads, but initiate mutate
flows only to RGs to handle writes.  AGs and RGs do not initiate any flows of
their own.

AGs are always the coordinators for the files they publish.  They mark their
files as read-only, and will not participate in any mutate flows for them.

When a UG wants to write, it does not publish

\subsubsection{Garbage Collection}

An interesting consequence of immutability is that writes to a record will cause
overwritten blocks and manifests to become unreferenced.  To prevent leaks, Syndicate's gateways
execute a distributed garbage-collection protocol to remove them.  The process
is asynchronous and tolerant of gateway failures.

When the coordinator of a record uploads new metadata to the MS, it includes a
vector of block IDs and the old manifest ID.  These are appended to a per-record
log in the MS.  Once the write completes, the coordinator asynchronously queries
the MS for the first $k$ entries in this log, constructs \texttt{delete}
requests for them, and sends the requests to the volume's replica gateways.
Once all replica gateways successfully acknowledge deletion, the coordinator
instructs the MS to remove the $k$ entries from the log.

\subsection{Metadata Service}

Syndicate's MS runs on top of a scalable NoSQL database.  In practice, our
deployments run within Google AppEngine~\cite{google-appengine}, meaning that
Syndicate's metadata is hosted in either Megastore~\cite{megastore} or
Spanner~\cite{spanner}.  In both cases, writes to a single key are atomic, and
multi-key transactions are allowed provided that the set of keys is small (e.g.
five or less in the implementation).

\subsection{Interfaces}

Syndicate has multiple front-ends that 


% --- notes

\section{Implementation Considerations}

By giving developers the freedom to implement storage processing at intermediate
points in the network, we enable them to design application storage that
respects each organization's hosting policies as data enters or leaves the
organization's doimain.  This is important for both scientific data and
decentralized applications, since they require coordination across
organizations.

\subsection{Example: Decentralized Document Editor}

For example, the user of a decentralized shared document editor
would want to ensure that other peoples' writes to her files are vetted
before being made externally visible.  To do so, she ensures that her volume's files
are coordinated by a single gateway running on her laptop.
Her volume's aggregation driver logic makes other
gateways buffer their Publish requests to one of a set of network-addressable
message queues, so that when
her laptop comes online, it will dequeue them and present them to her for
explicit acknowledgement.  This user works with confidential files at work and
personal files at home, so the Publish queue for work must be addressable only
on her employer's servers (the queue for her personal files may run anywhere and
be globally addressable).

Implementing this application without SDS would be
challenging, since the application would need to be aware of the fact
that there is a choice of which of the queues to use, and that the choice depends on
the file being edited.  With SDS, the
user only needs to load an aggregation driver for her volume that routes 
and mutate flows to the right queue.  Any application that can access her volume
automatically gets the required storage semantics.

\subsection{Example: Sharing an HPC Cluster}

As another example, a PI maintains an HPC cluster that her lab needs to share
with other collaborators.  Each collaborator hosts their datasets locally, and
needs to write back their changes from the HPC job.  The HPC cluster has a
``staging space'' and where collaborators may write their input data.  The
cluster itself has an ``output space'' where nodes write the results of their
computations.

Getting data to flow from a collaborator's lab to the HPC cluster and back is
straightforward with SDS.  The PI creates a volume for the staging space data
and a volume for the output space data.  Both volumes' aggregation drivers know
how to identify which collaborator issued which Publish or sent which chunk.
Each HPC node is given a read-only gateway to the staging space and a write-only
gateway to the output space.  Each collaborator is given a gateway for writing
data into the staging space, and a gateway for receiving data out of the output
space.  The aggregation driver logic for both volumes simply tracks which chunk
came from which collaborator, so the output space writes will be sent to the
right collaborator's gateways.

The result is that from the collaborator's perspective, submitting a job to the HPC
cluster is just a matter of writing the input into the local staging space
gateway, and later reading the output from the local output space gateway.  The
collaborators gets to choose which application-facing interfaces their local
gateways run, and get to use a reusable set of service libraries to push and pull
chunks into their storage providers and local storage facilities.
Without SDS, the PI would need to implement a job submission middleware that
handled collaborator-required application interfaces, collaborator-chosen
storage, and collaborator authentication.

\subsection{Gateway Roles}

Our examples show that gateways will assume specialized roles based on
which services they interact with.
While the SDS system gives the volume owner free reign over what each gateway is
capable of doing, in practice there are three common gateway roles.
They are:

\begin{itemize}
    \item A \textbf{replica gateway} is a gateway that runs service drivers for
cloud storage providers.  They do \emph{not} implement a storage programming
interface; they only interact with other gateways.  Moreover, these gateways are
not coordinators for any data (i.e. they never Publish).  Instead, other gateways send
them chunks and fetch them back.
    \item An \textbf{acquisition gateway} is a gateway that runs service drivers
for external data sets providers.  They do \emph{not} implement a storage
programming interface, but instead expose external data as read-only SDS data that happens
to be ``owned'' by the gateway's operator.  As such, these gateways are always
the coordinators for the data they expose.  However, they do not accept chunks.
    \item A \textbf{user gateway} is a gateway that runs service drivers for
CDNs.  Unlike the other two, these gateways \emph{do} implement a storage
programming interface, such as a filesystem or an HTTP RESTful API.
User gateways are the coordinators for the data created by the user that runs
them.  The name for this gateway is apt, since application users run
these gateways so their clients can interact with their data.
\end{itemize}

In our experience, we have found that these three roles are sufficient to
express complex application-specific storage semantics in non-trivial
applications.  Other roles are allowed by the SDS design parameters, but we have
not explored them.

In practice, the volume owner assigns each user-owned device a
user gateway, and gives it a service driver for fetching chunks via a 3rd party
CDN.  She creates one or more replica gateways owned by her SDS user account,
which will run service drivers for her volume's cloud storage providers.  These
replica gateways will simply load and store chunks to and from these services.
If the application needs access to
third party datasets, she will additionally create acquisition gateways for
herself that will index the 3rd party datasets (via dataset-specific service
drivers) and Publish SDS data records for them that map onto the dataset.


\chapter{Applications}
\label{chap:applications}

This chapter presents three applications built with Syndicate and Gaia.
In all cases, the ability to control end-to-end semantics within SDS
(instead of the application) enables
developers to tackle difficult data management
techniques, in ways that both preserve backwards-compatibility with existing
applications and preserve forward-compatibility with future storage features.
Applications do not need to be modified to leverage
new commodity services, and data flows and gateway placement let developers
consistently solve data management problems across multiple applications.

\section{Serverless Groupware}

% Groupware with Gaia
Groupware is a common category of Web application that allow users to
collaborate via data-sharing.  Groupware applications include shared to-do lists, calendars,
documents, contact lists, and so on.  Multiple users read and write to the same
storage medium in order to coordinate their activities.

The data storage story for groupware today requires each user to be able to see
a consistent view of her data, regardless of which of her devices read or write
it.  Since groupware is often used in sensitive
settings such as corporations, users have an expectation of privacy---by
default, their state is only visible to their devices.  Users must
\emph{explicitly} share data with other users (or the public), and if they do
so, their shared data is visible to all other users on all of their devices.

In conventional groupware software, this is achieved by running a shared server.
The users in the same user group have read and write access to the server's
state, and the server resolves conflicts between writes and enforces access
controls.  In addition, the server takes advantage of its global view of the
users' state to build up derived state like edit histories and
backups.  From a data policy perspective, all users trust one
organization composed of the server and all of the user groups'
devices.

In multi-organization settings, or in settings where users do not directly know
one another, implementing shared groupware is more challenging.  Each user (or
subgroups of users) have different policies regarding how their data is to be shared.
For example, a user's personal calendar should not be shared with work
colleagues.  What is needed is a groupware system where users can \emph{self-organize} into
user groups with which to share data, in a way where users can easily
authenticate one another and establish trust relationships with minimal
coordination.  This is achieved with a Gaia groupware library.

The groupware library differs from existing groupware software in two key ways.
First, it lets each user host their data on whichever cloud services (or
servers) they choose, while preserving end-to-end storage semantics for
groupware applications.  Second, it gives each user the ability to vet each
other user in the system by having users prove ownership of existing social
media accounts.  This latter feature allows users to self-organize into their
own per-application organizations with minimal coordination.  By posting
machine-checkable proofs-of-ownership on social media that are cryptographically
linked to accounts in Gaia's SSI system (henceforth referred to as ``social
proofs''), a user can easily vet other users when deciding to share groupware
data with them.  For example, users can leverage social proofs to prove that
they work in the same company, or go to the same school, or have the same shared
interests.

\subsection{Motivation}

Groupware software falls into two categories:  in-house groupware servers that
the users of an organization must maintain themselves, or outsources groupware
servers that run in third party servers.  There are undesirable 
trade-offs for both types of groupware.  In the first case, users incur an
ongoing operational cost for keeping the software up-to-date and keeping the
server running.  The advantage, however, is that they unilaterally control all
aspects of the server's data storage---including how often it gets backed up,
who can view the data, what kinds of derived state it makes, what version(s) of
the software it runs, and so on.

The second type of groupware is increasingly popular.  Companies like Microsoft
and Google each have suites of software-as-a-service offerings that take the
operational responsibilities out of the user's
hands~\cite{gapps}~\cite{microsoft-apps}.  The advantage is that the
SaaS offerings have potentially higher uptime and are managed by experts, and
are available at a predictable cost to users no matter how easy or hard it is to
maintain it.  The downside, however, is that the SaaS provider has global
visibility into the users' data, regardless of the users' desired privacy
settings.  If the SaaS provider is hacked, their groupware data can be exposed
to the public.  If the SaaS provider goes out of business, the groupware data
can be lost forever.  If the SaaS provider changes its API, then any custom
integrations with the platform break.

There does not exist a middle ground where users can share their data in a way
that is convenient for them (like what SaaS offers), but with the policy
controls they would get by running an in-house groupware server.  The serverless
groupware library for Gaia fulfills this need.

\subsection{Role of SDS}

Gaia enables the best of both worlds.  Users get all of the
operational convenience of SaaS with the privacy and data controls of having
their own servers.  Importantly, Gaia allows users to select whichever storage
providers they want without affecting the design of the groupware software. 
In addition, ancillary functionality like search indexing can be
implemented in Gaia gateways and reused in other applications by way of the
global relational database design pattern described in the previous chapter.

The users rely in Gaia's SSI system to bootstrap data confidentiality and
authenticity.  The gateways in Gaia ensure that all data is signed and encrypted
when it leaves the device, such that only the user's designated recipients (if
any) can view it.  In addition, the groupware software uses Gaia to ensure that
applications are isolated from one another at the volume level---an application
client can only access application-specific state.

A key operational concern of groupware systems is that they must only allow
users to view one another's data \emph{with the owner's permission}.  Gaia's
gateways enable this by allow users to implement data-specific checks when sharing
data.  This is achieved by giving users the ability to create and vet one
another's social proofs.  Importantly, the social proofs are verified
automatically by the software and presented to the user as part of the
permission-granting user experience.

\subsection{Design}

The groupware software is designed to run within the Web browser.  The
application logic runs as a Web page, and loads and stores the user's
credentials and data via a co-located Gaia node.  This allows decouples the
user experience and application functionality from the user's shared
storage concerns.  For example, one user can store their data on Dropbox and
another user can store theirs on Google Drive, but the application can access
each user's data regardless via the Gaia node.  A system overview is given
in Figure~\ref{fig:chap4-gaia-groupware}.

\begin{figure}[h]
   \centering
   \includegraphics[width=0.9\textwidth,page=23]{figures/dissertation-figures}
   \caption{Design of serverless groupware with the Gaia SDS system.  Alice
   lists signed certificate graphs in her SSI user account data, as well as the
   list of her personal devices' public keys and social proofs.  While Alice can
   write to her storage from her private Gaia nodes, she can make her data
   available via a public Gaia node as long as her SSI account contains enough
   social proofs that she is a valid application user.  Bob
   uses this public gateway to discover and read her shared data.}
   \label{fig:chap4-gaia-groupware}
\end{figure}

\subsubsection{Setup}

A user receives a volume for each groupware application she uses.  When she signs up for a specific
application, the groupware software inserts an application-specific set of keys
into the user's SSI account information, indexed under the application's name.  To provide
confidentiality, the user has the option of encrypting this routing information
such that only her trusted peers can discover that she uses it.  Her other
devices and other users' devices inspect her account in the SSI system to determine
which keys to use to authenticate the data she writes, as well as discover how
to access her storage (i.e. which Gaia nodes to contact, which storage providers
to contact, etc.).

\subsubsection{Sign-in}

The groupware software employs device-specific keypairs to allow the user to
sign in via multiple devices.  When the user signs in for the first time, her
device creates a volume for her and registers \emph{all} of her devices as
belonging to the same volume owner.  Then, when the user signs in from a different device, she can
still read and write data to her existing volume and administrate it.

The software ensures that her devices are aware of each other via a
``delegation record'' in her SSI account.  The delegation record lists all of
the user's device IDs and their public keys.  This way, when the user creates a
new volume, the software automatically grants all devices the volume owner
privileges.  To the user, it appears that they simply began using the app from a
separate device, just as they would have had it been a conventional Web
groupware application.

If the user wants to add or remove a device, she must re-generate her delegation
record with the current set of device public keys.
To do this securely, the software requires a quorum of signatures from a trusted
subset of her devices (configurable by the user).
A delegation record will only be considered valid if it
is accompanied by a sufficient number of signatures from this trusted device
set.  For example, a user might require a signature from two of three of her
devices in order to add a fourth device or remove the third, and in doing so
tolerate the loss of one of her three devices.  This way, the user can control
which devices are allowed to write to her data while tolerating the loss or
compromise of a pre-configured set of them.

Both the quorum threshold and the public keys of the
trusted devices are listed in the user's SSI zone file.  Since changing the zone
file requires a blockchain transaction in the SSI system, there will be a
widely-replicated auditable log of each user's device key rotations.  This makes it
easy for users (and their collaborators) to check key lifetimes, and makes it
risky for attackers to attempt to change keys (since they cannot do so
silently).

When the user signs in, the groupware library creates a gateway for the device
she is using if one does not exist already.  Her device will sign the
new certificate graph for the app's volume and make it available in her SSI account.  The
software authenticates data from the user by (1) looking up the user's ID in the
SSI system, (2) extracting the trusted device public keys and quorum threshold
from the zone file, (3) validating the delegation record, and (4) validating the
certificate graph against the delegation record.  The software caches
monotonically-increasing version numbers for the certificate graphs locally to
prevent stale certificate graphs from being reused.

\subsubsection{Reading and Writing Data}

Since a user gives each application its own volume, a groupware application like
a shared calendar spans the set of users' devices.  Gaia ensures that when the
application client is loaded, it only has visibility into the
application-specific volumes the users have created (i.e. so a malicious
or buggy application cannot read another application's state).

The groupware storage interface references data by its volume key and owner user.  For
example, to read Bob's file \texttt{today.cal}, Alice's application client would call
\texttt{get(``today.cal'', ``bob.id'')}, where \texttt{bob.id} is Bob's username
in the underlying SSI system.  All the while, Gaia
ensures that Alice's calendar application only discovers the routing information 
to Bob's calendar volume.

\hfill \break
\noindent{\textbf{Read Authorization}}
\hfill \break

When writing shared data, the user must ensure that it is readable by a given
set of other users.  How does the writer identify these other users,
and how can the software identify users as belonging to particular
organizations?  The groupware library addresses these problems by both allowing
the writer to specify other individual readers, and by allowing the writer
to specify which social proofs a reader must have (as well as a way to vet
them).  

The user is free to choose which proofs are required for their
application, depending on the application.  For example, a cryptocurrency
investment application could require a user to produce a signed KYC
(know-your-customer) attestations
from the government and the user's bank that prove that the user is an
accredited investor.  This proof would be signed and stored in a social media
platform that the groupware library can crawl (such as
AngelList~\cite{angellist}).

Once a Gaia gateway knows which social media proofs are required to read
a key value, it will only accept read requests from users who present
the requisite proofs.  To facilitate this check, users insert URLs to the
proofs within their SSI account linked to their names in the SSI system (which
the Gaia gateway looks up on-the-fly).

\hfill \break
\noindent{\textbf{Searching}}
\hfill \break

Public groupware data is readily indexed by anyone who wishes to stand up a
Gaia database instance to crawl the set of application-specific
volumes. In addition, private groupware data can still be indexed---either by a
trusted, private Gaia database, or by downstream user groups.

To implement private search in a user group,
the groupware software ensures that the local device's Push
stage indexes the contents of the file before encrypting and replicating it.
The Push stage encrypts the index data with the viewers' public keys, so
the viewers will be able to search for the file by keyword.

The index itself is application-specific, but can do things such as
associate search terms to file names and word counts.
The index data is structured as a per-user prefix tree, so
that a search query only needs to fetch a narrow subset of the index to find
files with the search term.

A global untrusted relational database can accelerate delivery of encrypted
index files to downstream readers.  Trusted readers asynchronously fetch,
decrypt, and incrementally reconstruct the writer's index locally to service
search queries.  Depending on the sizes of the index and the number of users,
the application may take different strategies for fetching the encrypted
index---for example, a large user group may employ a private trusted instance of
a Gaia relational database that can eagerly build up a search index, whereas a
small user group may simply fetch and reconstruct each other users' indexes as
needed.

\subsection{Implementation}

The groupware library implementation is the work of multiple contributors.
It is implemented in two parts: Javascript library that facilitates user
sign-ins and application-specific volume creation, discovery, reads, and writes,
and a UI that allows users to manage their social proofs.
It was developed in collaboration with Blockstack Public Benefit Corporation~\cite{blockstack-pbc}.

Several applications have been independently built by Blockstack community members
with the groupware library.  Examples include:

\begin{itemize}
   \item \textbf{Blockstack To-Dos}:  This is a private to-do list application
      that uses single-reader Gaia volumes to store private user to-do lists.
   \item \textbf{Graphite}:  This is a Google Docs work-alike~\cite{graphite-docs}.  Users store and share documents and
      spreadsheets via multi-reader Gaia volumes.  The data is encrypted by
      default, so that only the designated readers can access it.  It makes use
      of a Gaia database to facilitate secure document discovery---the database
      discovers encrypted pointers to the encrypted document, so that only the
      intended recipient can access the data.  It also offers end-to-end
      encrypted messaging, where messages are replicated to Gaia volumes for
      long-term storage.
  \item \textbf{Blockstagram}:  This is an Instagram work-alike that allows
     users to securely share photos via multi-reader Gaia
      volumes~\cite{blockstagram}.  Photos are
     encrypted with the recipients' public keys before being replicated, thereby
      providing end-to-end confidentiality.  It was developed by a team of eight
      Web application developers with no prior experience with Gaia (or
      Blockstack, Gaia's SSI system) in less than 36 hours at a hackathon in
      Berlin~\cite{patrick-tweet-blockstagram}.  % https://twitter.com/PatrickWStanley/status/970307376690626561
  \item \textbf{Stealthy.im}:  This is an end-to-end encrypted chat application,
     where users can securely send text and pictures
      real-time~\cite{stealthy.im}.  It uses
      multi-reader Gaia volumes to store chat data, and uses a Gaia database to
      discover and invite users to chat.  A similar Gaia-powered application is
      \textbf{Hermes}~\cite{hi-hermes}.
  \item \textbf{Coins}:  This is a private cryptocurrency portfolio application
     that uses single-reader Gaia volumes to securely and confidentially store
      the user's cryptocurrency holdings~\cite{coins}.  It allows the user to track the worth
      of their holdings without exposing them to anyone outside of the user's
      computer.
  \item \textbf{Publik}:  This is a microblogging application that uses
     multi-reader Gaia volumes to share blog posts~\cite{publik}.  A Gaia
      database for indexing hashtags and user posts is under development.
  \item \textbf{Bellweathr}:  This is a business analytics program that uses
     machine learning in the user's Web browser to help a business owner
      identify patterns in customer purchases~\cite{bellweathr}.  Business
      owners use Gaia to load and store encrypted copies of their customer data
      and trained models, thereby ensuring that it will remain private.
      Equivalent applications today require business owners to expose their customer
      data to third parties, which puts both they and their customers at risk 
      to hackers and security mishaps.
\end{itemize}

All of these applications use Gaia and its SSI system to load, store, and share
user data.  The SSI system implementation (the Blockstack Naming
Service~\cite{bns}) removes the need for per-app password databases and per-app
identity services, and Gaia removes the need for per-app data silos.  Users can
share data from one application to
another~\cite{blockstack-technical-faq-share-data} without the application's
permission or cooperation.

The applications Graphite, Blockstagram, Stealthy.im, and Hermes all rely on a
global database instance to discover other application users.  They are not
coupled to a specific instance; anyone can deploy a new global database if the
default instance misbehaves or is not trusted.

\subsection{Discussion}

The usefulness of SDS is apparent in its ability to implement its users'
data-hosting policies independently of the applications.  Each user can keep their groupware data on
the storage providers of their choice, and in doing so, control their
availability, durability, and access control independently of one another and independently of
the applications.  For example, a user's Gaia node can programmatically delete
old Stealthy.im messages without Stealthy.im's permission.  As another example,
a user's Gaia node can limit access to its owner's Graphite documents by denying
reads from hosts outside its local area network.

At the same time, application developers do not need to care
about hosting user data, and do not need to worry about coupling their data to
specific storage systems.  All of the third-party applications above do not rely
on application servers.

As an optimization, their respective developers deploy
Gaia global databases to help users discover one another.  For example,
Stealthy.im implements an invite mechanism using a Gaia global database, and
Graphite uses a Gaia global database to help users discover shared files.
However, the developer is not required to deploy and maintain a global database.
Gaia global databases only host soft-state in the application, and any user can
instantiate their own global database and derive the same 
database state.  This means that as long as at least one user is interested in
preserving Stealthy.im's invite system or Graphite's
document discovery system, they can do so without the developer's help.

The expressive power given to developers by the aggregation driver model is
apparent in the ability to control read and write access based on whether or not
the requesting user has made particular social proofs.  The social proof check
code only needed to be written once, and it now works across all groupware
applications and all cloud services.  The expressive power is also apparent in
the ability to automatically generate private search indexes in response to reads and
writes.

The main difficulty with giving users direct control over their groupware data
today is that it forced them to run a shared groupware server (or collectively
trust someone to do so on their behalf).  By instead
implementing what used to be server-side functionality as aggregation driver
stages, the library removed the need for a shared server while preserving each
user's control over their data.

\section{End-to-End Encrypted Email}

The ability for SDS systems to instantiate application-specific data flows gives
users the power to enforce data transmission and storage concerns in
\emph{existing} protocols as well.  This is demonstrated by using Syndicate to construct
end-to-end encrypted email that addresses long-standing
usability concerns that impede the widespread use of PGP~\cite{pgp}.

\subsection{Motivation}

Encrypted email is not a new concept.  However, it has proven notoriously difficult to
deploy~\cite{why-johnny-cant-encrypt}
~\cite{why-johnny-still-still-cant-encrypt} due to the need for users to manage
private keys.  In addition, deploying end-to-end encrypted email over legacy
SMTP servers and clients leaves users vulnerable to two security flaws:  users
can only achieve end-to-end encryption if they all share keys, and users can
accidentally leak other users' cleartext when including new users in an email
thread.

\subsubsection{Using Private Keys}

Even if users had a good understanding of public key cryptography, they must still contend
with key distribution and key revocation.  Key distribution is not addressed by
the encrypted email systems studied.  However, existing methods---key escrows,
certificate authorities (e.g. S/MIME~\cite{smime}, DANE~\cite{dane},
x.509~\cite{x509}), and webs-of-trust are difficult to use securely, and easy to
use incorrectly.

Key escrows and certificate authorities are ``centralized''
entities that often live outside of a users' organizations, which makes it
difficult for users to reason about their trustworthiness.  Only organizations
whose data policies admit a trusted third party can make use of these services.
Trusting a third party for such a task carries the risk of compromise: if a
widely-used certificate authority is compromised, it can lead to widespread
data exposure.  Users may not discover until after harm has been done to
them, such as identity theft.

Webs of trust do a better job than centralized key servers at preserving
organizational autonomy because they allow each organization to unilaterally
decide which other organizations to trust.  However, there is a high
coordination cost in maintaining them.  This
is because trust is \emph{not} transitive by nature---if Alice trusts Bob and Bob
trusts Charlie, it does not follow that Alice trusts Charlie.  Users in each
organization need to be wary of the degree to which to trust their peers, and wary of the trust
judgments their peers will make.  Moreover, they must curate their webs of
trust to account for changes in the organization.  For example, if Bob is fired
from his job, then all of Bob's coworkers must update their webs of trust to stop trusting his
email signing key.

Key revocation adds another layer of complexity.  Key revocation certificates
and signed key expiration dates do not go far enough in making encrypted email
usable.  If a user loses both their private key and their key revocation
certificate, then they have to get other users to re-establish trust in them
from scratch.  If the user's private key is compromised, then the attacker can
send arbitrary emails before the user can transmit their key revocation
certificate.  If the user loses their revocation certificate, or if the attacker
can stop the certificate from reaching the victims, then the user cannot stop an
attacker with their compromised private key.

\subsubsection{Contacting other Users}

Even if users could reliably distribute and revoke public keys, conventional
email clients still allow users to communicate with others in insecure ways.
Users can bring harm to themselves by accidentally sending email in the clear
when they meant to encrypt it.  Also, users can bring harm to others
by accidentally divulging their communications by carbon copying
their cleartext in an email to a user who does not use encryption.

Neither existing SMTP clients (including Web clients) nor
SMTP servers address these problems.  SMTP clients do not help users with key
distribution or revocation, and they do not help the user discover whether or
not they have the right key.  Web SMTP clients are even less secure, because the
Web client offloads transmission to a remote server (which now must be trusted
by the user).  If the user wants to use another device to send an email, such as
a public terminal, they have to divulge a private key to the device.

SMTP is already ill-suited for encrypted communications because at a minimum the email's
sender and recipient must be readable by all SMTP servers between the sender and
recipients.  Also, due to its store-and-forward architecture, any messages
accidentally sent in the clear will be stored by the servers for an
indefinite amount of time.  Users do not get to choose which servers store and forward
messages, and users cannot ``unsend'' messages if they discover that they sent
them to the wrong recipient.

\subsection{Role of SDS}

This thesis presents a backwards-compatible mail system built on top of
Syndicate.  Unlike conventional email, the Syndicate email system
automatically encrypts data end-to-end and ensures that users
discover each other's \emph{current} public keys by way of its SSI
system.  User can do the following with this system:

\begin{itemize}
\item \textbf{Automate key management}.  Users do not need to interact with keys
at all.  Users do not need to trust external key escrows or certificate
authorities, and they do not need to participate in webs of trust.  Instead,
users rely on Syndicate's blockchain-powered SSI system to discover each other's
current public keys.

\item \textbf{Control where emails are hosted and who can request them}.
A user's message contents will
not be relayed through the SMTP network, but will instead be hosted in one or
more storage hosts of the user's choosing.  Recipients will instead download and
decrypt the message once they have discovered where it is hosted and have
obtained sufficient permission.

\item \textbf{Support sending to legacy users}.  The Syndicate email system does \emph{not}
require both sender and recipient to use the same client in order to achieve
better security than legacy email.  If the recipient does not use this new system, the sender has
the ability to contact the receiver while
preserving sender-chosen security properties.  For example, 
the sender can share the message body via a trusted private shared cloud storage folder
that only the sender and receiver can access, and send the URL to the message
body via SMTP.  Only the recipient will be able to access the data.

\item \textbf{Safely use untrusted devices}.  This secure email system uses Syndicate's SSI system to
allow users to derive short-lived throw-away keys for signing and encrypting
messages on untrusted devices, like public terminals.  The keys are
automatically distributed and revoked.
\end{itemize}

\subsection{Design}

The Syndicate email system follows a similar design to the Internet Mail
2000~\cite{internet-mail-2000} proposal.  Users store their
encrypted emails in a Syndicate volume, which they
use to selectively give recipients access to their messages.  The system uses
the SMTP network to allow senders to inform receivers when they have new
messages waiting for them (Figure~\ref{fig:chap4-syndicate-mail}).

\begin{figure}[h]
   \centering
   \includegraphics[width=0.9\textwidth,page=24]{figures/dissertation-figures}
   \caption{Design of end-to-end encrypted email with Syndicate SDS.  Alice can
   send email from both a personal device and a public terminal; the latter of
   which gets assigned a temporary session key that expires shortly after being
   created.  Bob's client detects new mail from Alice via the legacy SMTP
   network by receiving a signed list of URLs that point to Alice's chosen
   storage services.  If Alice emails non-users of this system, her UG employs a
   custom ``message gateway'' (MG) type to Push the message payload to them
   while enforcing her custom security policies (such as ``store this message in
   a private shared Dropbox folder that the recipients can access and email them
   the URL'').}
   \label{fig:chap4-syndicate-mail}
\end{figure}

\subsubsection{Setup}

Each user stores their preferred email address in the SSI system.
Alice sends a message to Bob by looking up Bob's account
information in the SSI system, and then obtaining his email address.  In order
to convince Alice that he is the ``right'' Bob (i.e. the Bob she is looking
for), he includes additional credentials in his SSI data, such as social proofs
or signed attestations from trusted third parties.  The Syndicate email system
is not concerned with implementing a particular authentication strategy, but instead gives users
the ability to prove that various pieces of user-submitted identifying state associated with
the email address are signed by the same key that owns the email address
in the SSI system.  For example, if Alice knows that Bob owns the website
\texttt{www.bob.com}, Bob could authenticate to Alice by hosting his SSI
username and a signature on
\texttt{www.bob.com} and list a pointer to \texttt{www.bob.com} in his user
account on the SSI system.

The system is designed to accommodate multiple devices owned by the user by
storing all emails in a single volume that spans the user's devices.  Each
device has its own key-pair in the volume certificate graph, which is used to create
gateways specific to that device.  The user has an ``admin'' email account (i.e. an
account that is tied to the Syndicate volume owner account that stores her
emails).  The admin account is controlled
from a trusted device and is used to add or revoke permission to communicate from
other devices.

When a user signs up for the system for the first time, she downloads and
installs a mailer daemon
that implements an SMTP and IMAP endpoint locally.  The user points their preferred email
client to the local mailer daemon to send and receive messages.  In addition,
the daemon implements an HTTP interface for serving the mail client encrypted
messages from the Syndicate volume.

The mailer daemon prompts the user to generate a device-specific
Syndicate user account and two gateways (a UG and an RG)
when it is installed.  The user does so by using her
admin account.  The installer wizard gives the user the option of pre-allocating
keys for her devices and their gateways, which can be fetched and installed on untrusted devices
on-the-fly without requiring her to use her admin account again.  Their
keypairs are encrypted with a password of the user's choice, and stored to the user's
volume.

\subsubsection{Signing In}

Each device the user sends mail from receives its own keypair.  Each
device-specific key is associated with an optional expiry timestamp and
revocation certificate, which are stored in the user's Syndicate volume for
safekeeping.

Signing in with a new device requires ensuring that the device-specific private key is
available.  For devices the users trust, this is achieved simply by (1)
installing the software, and (2) allowing the device to register its public key
with the user's account in the SSI system.  An untrusted device, such as a
public kiosk, would receive a key with an expiry date and revocation certificate.
When the user signs out of the device, she would ``activate'' the revocation
certificate by appending a signed timestamp to it and 
moving it to a canonical path in her volume.  Other users'
clients would discover and process it automatically when receiving a message,
thereby ensuring that the kiosk does not use the private key after the user is
done with it.  The key expiry timestamp
ensures that the key expires nonetheless if the user is unable to successfully
sign out (i.e. unable to post the revocation certificate).

The device-specific key state includes the device-specific user account and the
device-specific gateway keys that the mailer daemon will use to interact with
the volume.  Each devices' gateways only write to one directory of the volume,
and mark their files as read-only by other devices (which the MS enforces).
The mailer daemon develops a coherent view of the mailboxes by listing all of
the devices' directory states.

In addition to creating device-specific Syndicate keys, the software also
creates a generic read-only UG and read-only RG whose private keys are publicly readable
and exposed in the volume.  These gateways are meant to allow recipients to access
the volume's ciphertext, so the designated recipient can decrypt them.
They are configured in the certificate graph to only have read capabilities, and
to only serve on \texttt{localhost}.  This ensures that all of the user's other
gateways will ignore them, and that anyone can run them on their computers to
access the inbox data.

\subsubsection{Sending and Receiving Mail}

The mailer daemon implements a Syndicate UG and RG (e.g. as subprocesses).
The UG implements the SMTP
and HTTP endpoints, and the RG uploads messages to the user's preferred storage
service, such as their personal Dropbox folder or a S3 bucket.

When the UG receives an outgoing message, its \texttt{serialize()} driver method inspects the
message for the recipient, and automatically looks up the public key in the SSI
system to encrypt the message to the recipient before sending it to the RG.
\emph{This way, the sender is never involved with selecting the key for a
recipient user}.  The software additionally makes a copy of the sent message encrypted with the
sender's public key, and stores it into the device's ``sent'' mailbox.

The mailer daemon informs the recipient that they have a message waiting for
them by sending a small amount of discovery information to the recipient's
email address via SMTP.  This discovery information is signed by the sender,
to prove its authenticity to the recipient.  It identifies the path to the
message in the volume, as well as the hash of the ciphertext.

The recipient's mailer daemon polls the user's SMTP inbox for discovery
messages.  When it finds one, it fetches, authenticates, and decrypts the
associated message from the sender's volume,
and locally stores it so the user's mail client can read it as a normal
email.  It does so automatically as part of the \texttt{deserialize()} driver
method in the UG---this driver method only succeeds if the message could be
authenticated.  The discovery message's sender email address
is used to look up the user's device keys in the SSI system to perform the authentication.
\emph{This way, the receiver never needs to select the key for the
sender to authenticate the message}.

The sender must host the email contents for the recipient until either the
recipient downloads it.  Once the recipient daemon has fetched the cleartext, he
encrypts and backs up a copy via its RG for safe-keeping.
The sender can delete the messages she sent at any time, thereby granting her
the opportunity to ``un-send'' an email's message body if she can do so before
the recipient fetches it.  The sender can garbage-collect old messages once
she is sure the recipient has fetched them, or once the information is no longer
relevant.  For example, the sender could simply delete all messages she
sent over one month ago.

If the sender includes multiple recipients, or includes a new recipient part-way
through the email chain, their mailer daemon detects this and ensures
that the previous conversation is kept secret.  This is achieved by having the
local RG in the mailer daemon remember which email threads have which
recipients, and ensure that their respective messages are re-encrypted before transmission.
This conversation metadata is encrypted and stored on the user's volume, so it
is accessible from all devices' RGs.  This decreases the likelihood that a user
accidentally divulges cleartext in carbon copies on the email client---the message
would simply fail to send if the user did this.

\subsubsection{Legacy Compatibility}

As with PGP before it, the Syndicate-powered email system requires both sender
and recipient to use it in order to realize the full benefits.  Unlike PGP, the
developer can ensure that certain safety features are in place if only the
sender uses the software.  This is made possible by Syndicate's
aggregation driver programming model.

It is important to recognize that when it comes to email, the correct way to
send a message depends on the sender, the recipient, the content, and the context in which
it is sent.  For example, two friends exchanging vacation photos do not need the same
security guarantees as an anonymous informant communicating with a law
enforcement agent.

One of the major drawbacks of PGP is that cannot work if either the
sender or recipient do not use it.  This significantly limits the set of senders
and recipients.  Moreover, PGP-encrypted messages are easy to spot in SMTP
traffic, which makes it easy for network eavesdroppers to identify users who
have something to hide.

What is needed is for senders and receivers to be able to communicate even if
one of them does not use PGP-like encryption.  The approach taken here is to
make it easy for the sender to control how the message will be delivered, while
allowing messages to be discovered by the recipient over legacy SMTP.  The
sender is free to set up the delivery process to implement the security
guarantees on a case-by-case basis, subject to what she knows about the recipient and subject to
the contents of the message.  For example:

\begin{itemize}
\item The sender can encrypt the message with a password known to the recipient,
and send the message body in a common document format, like Microsoft Word or
PDF, that the recipient can open and decrypt with already-installed software.
This can provide the confidentiality of PGP.
\item The sender can replicate the message to a shared private storage provider
like a Dropbox folder or private \texttt{git} repository, and send the
recipient the URL over SMTP.  This process can be carried out via HTTPS.
While this does not provide the same
degree of end-to-end confidentiality and authenticity as PGP, it guarantees that as long as
the certificate authorities and shared storage are trusted, then only the sender, the recipient, and the
storage provider can view the message (but SMTP servers see nothing).
\item The sender can select which network to use to transmit the data, based on
the recipient.  For example, an enterprise user could require all messages sent
to the company SMTP server must be sent through the corporate
VPN.  The aggregation driver would refuse to send messages unless it detected
that the VPN was available.  This ensures that all email messages sent by employees are
visible only to the company and the recipient.
\end{itemize}

These examples do not provide the same guarantees of PGP, but they are
better than relying only on legacy SMTP for email.  While they can all be done
today in an ad-hoc manner without SDS today,
Syndicate lets users ensure that they are all executed
automatically and consistently.  Moreover, the way these features are
implemented allows them to be reused in multiple different contexts, giving senders the
ability to \emph{combine} different features to create a custom message
transmission process.

Addressing legacy compatibility is a practical application of Syndicate's custom
gateway types.  The deployment designed so that the RG's
Push driver stage (1) reassembles the Pushed chunks received from the UG
(embedded in the email client) back into the original email, (2)
scans the certificate graph for gateways with a type identifier specific to the
email client (the ``MG'' gateway in Figure~\ref{fig:chap4-syndicate-mail}),
and (3) forwards the reassembled email to them for further
processing.

When the MG receives the message, it inspects the message
headers and runs a user-specified program based on the recipient address.  The
user-specified program is responsible for actually transmitting the email.
For example, each of the above examples can be implemented with separate
programs that are invoked as subprocesses that take the message as input and
carry out the actual transmission.

The transmission programs themselves are part of the email-type gateway's driver.  The user
deploys them to her volume by updating the certificate graph.  Since the volume
spans all of her devices, each of her devices will have the most up-to-date
transmission programs available whenever the user sends a message.

\subsubsection{Search Indexing}

Since all messages are encrypted client-side, there is no option for server-side
message indexing.  Instead, the user's RGs incrementally build up a
word-to-email index as part of their Push stage logic, just as they do in the
serverless groupware example.  The index itself is
encrypted with the user's public keys, so it is visible only on the user's devices. 
In fact, the code to maintain the users' indexes can simply be re-used by the
RGs without affecting the design or implementation of the mail clients.

There are two reasons to offload search indexing to the RGs instead of allowing
applications to handle this.  First, this preserves the index across all devices. 
This is especially important for Web clients, which cannot easily store a large amount of state
locally on their own (HTML \texttt{localStorage} is limited to 5MB, for example).  Second, it
makes it easier to implement additional features like spam filtering, described below.

\subsubsection{Spam Filtering}

A key usability problem with encrypted email is that the servers cannot filter spam,
since they cannot read the messages.  This can be addressed in four ways within
the volume's aggregation driver.
\\
\noindent{\textbf{Shared Spam Database}}. First, the aggregation driver is programmed to have the RGs in a user's volume build
up a \emph{shared} set of classification data from user input.  When the user
moves data to the ``spam'' mailbox, the RG driver's Push stage generates and
a feature vector from the cleartext and stores it in a shared storage
provider.  This allows
users to share each others' spam feature information.

The shared storage itself is implemented as a separate, third party volume that enforces write-once read-many
access patterns, and tracks which users add which features.  That is, the RGs to the volume do not allow a record to be
written more than once, and do not allow records to be deleted (except by the
volume owner).  This ensures
that users do not accidentally clobber one another's writes, and a malicious
user (such as a spammer) cannot erase the feature vectors.  If it is later
discovered that a particular user's records were written with malicious intent,
they can be removed by the volume owner.

This arrangement is similar to existing third party spam detectors such as
Spamhaus~\cite{spamhaus}, where a third party aggregates spam information 
on behalf of many users.  The spam volume owner would aggregate the spam
information to train a spam classifier, and write the classifier parameters
to the volume.  A user's mailer daemon would connect to the volume in a read-only fashion
to read the classifier parameters, and use them to classify the user's inbound
messages as spam or not spam.  Because the volume is shared across many users
(and can be replicated by any user), the users are able to avoid spam-detection
service lock-in because they can (1) independently calculate the spam classifier
parameters, and (2) come up with their own, better classification system if the
spam volume owner does not do a good enough job.

Anyone can set up and run a collective spam filtering process.  Users are free
to unilaterally decide which ones to use.  Therefore, this approach does not
infringe on organizational autonomy.
\\
\noindent{\textbf{Sender Pays for Storage}}. The second anti-spam feature is that by design, the user pays for storing messages to recipients.  Since each
recipient has a different public key, the user must encrypt a message for each
recipient.  As a result, a spammer
must store a lot of state to spam many users at their own expense.  This
discourages, but does not completely remove, bulk spam.  This is similar to
the Internet 2000~\cite{internet-mail-2000} webmail proposal.
\\
\noindent{\textbf{SSI Proofs of Payment}}. The third measure is to take advantage of the fact that the SSI system is implemented on top of
a public blockchain.  This feature allows for some interesting anti-spam mechanisms.  A recipient can require the sender
to include a ``proof-of-payment'' on the message, generated by a transaction on the
underlying blockchain.  This would have the effect of both rate-limiting
spammers and making emailing users prohibitively expensive to do at scale.  It
would also allow senders to prioritize messages by paying higher fees. This
is a technique that was successfully employed by
Earn~\cite{earn-co}, for example, whereby a user will only see a
message if the sender has paid a minimum amount of money required by the
recipient.
\\
\noindent{\textbf{SSI Social Proofs}}. The fourth measure is to re-use a concept from Gaia-powered groupware to require
that a sender provide sufficient proofs in the SSI system that they are a
legitimate human being, and not a bot.  For example, a recipient can enforce a
default anti-spam policy whereby a sender must supply evidence in their SSI
account that they own at least five unique social media accounts, and that the
accounts undergo a minimum amount of activity.  This makes it hard to send
spam at scale because (1) the spammer would need to circumvent all of the social
media systems' anti-bot mitigations, and (2) if the spammer gets caught, they
have to register a new identity in the SSI system (necessitating a blockchain
transaction).  Since the blockchain itself grows at a fixed rate, and since
blockchain peers effectively bid on the ability to write new transactions, a
spammer could not easily register many identities without paying a high price
(i.e. the price gets higher the faster the spammer tries to register new
identities).  This allows the system to overcome the limits of prior proof-of-work
techniques~\cite{anti-spam-proof-of-work} which either had a fixed proof-of-work
threshold or a threshold that increased independently of the system's usage.

All four of these techniques would be implemented in part by the Acquire stage of the
mailer daemon's RG.  This ensures that all email clients automatically benefit
from these mechanisms without modification.

\subsection{Implementation}

The prototype system, SyndicateMail, is implemented in 4100 lines of Python and
1700 lines of Java.  It implements end-to-end encryption across multiple devices
and offers legacy compatibility with SMTP.

The system is being refactored to use the search indexing
logic from Gaia to implement search indexing in Syndicate.  The RG
driver runs the indexing logic as a subprocess in a \texttt{node.js} VM.  The
spam filtering is carried out simply by passing the text through an existing
spam-detecting system such as \texttt{spamd}~\cite{spamd} or \texttt{spam-assassin}
~\cite{spam-assassin}, and only forwarding the email text if it is not spam.

\subsection{Discussion}

In terms of the number of patches to write, it would be costly to implement this email system without
SDS.  Each email client would need to be patched to store its state to the
storage provider of the user's choice, whereas the use of Syndicate ensures that
storage services only need to be ported once.  By moving data signing,
encryption, decryption, and verification to the storage layer, 
and using Syndicate's SSI system to bootstrap key trust, Syndicate enables the
use of existing email clients with encrypted email without forcing
users to understand public-key cryptography.  By using gateways to represent the
capabilities of each device, the system is able to provide the convenience users expect from Webmail
without forcing them to manually copy private keys between devices.

Filtering spam and preventing accidental cleartext disclosure are problems
that require the system to inspect email contents on the user's behalf.  This
is achieved by having the user's RGs carry out this inspection locally,
instead of forcing the user to trust an external SMTP server to do so on their
behalf.  This is crucial to ensuring end-to-end message confidentiality, and
is required to be implemented at a layer \emph{beneath} email clients to ensure
that the user's choice of client does not alter the system's ability to ensure
message confidentiality and prevent spam delivery.  These problems
are both addressed by allowing the user to run application-specific aggregation driver
stages interposed between their personal devices and the rest of the network.

\section{CDN-accelerated Scientific Data Staging}

Scientific computing is increasingly conducted across multiple research groups.  Data is
generated and stored in the labs where a scientific instrument or dataset is
curated, and then shared across the world with collaborators.  Similarly,
collaborator labs publish their data analyses, which get downloaded by other labs
(and classrooms) for further consumption.

The third application presented here is to use 
Syndicate to implement a cross-site data processing framework that
allows scientists to take advantage of commodity cloud storage and CDNs to host
and deliver data to each lab.  For dataset curators, this reduces the task of
exposing a dataset to collaborators to running a Syndicate AG that can crawl the
dataset (with a dataset-specific driver) and serve chunks of it to downstream
UGs.  For dataset readers, this reduces the task of accessing a dataset to
fetching a dataset-specific Docker~\cite{docker} image that mounts the dataset
as a read/write filesystem backed by the dataset AG, an intermediate CDN, and
the user's personal cloud storage.

\subsection{Motivation}

The main motivation for considering an SDS approach to scientific data storage is
that due to the nature of the data they gather, each lab will have its own data curation
policies, its own unique data access patterns, and its own data-sharing policies.
There is not a one-size-fits-all approach for hosting scientific data, and labs will need to tailor
their storage systems to meet their specific needs (especially since their needs
change over time, depending on the nature of the data they produce).

This need to accommodate changing data storage and access policies is evident in
the evolution and wide success of state-of-the-art scientific storage systems
like iRODS~\cite{irods}, which offer
user-programmable policies (``rules'' and ``microservices'') that allow
individual scientists, project teams, and entire labs to programmatically
specify their curation policies and have them automatically enforced.
In fact, iRODS is considered to pioneer SDS concepts
(Chapter~\ref{chap:related-work}).

The scientific data-sharing framework uses commodity CDNs and cloud storage to help iRODS deployments
handle ``fan-out'' data distribution cases, where many labs across the wide
area want to read existing datasets and write back changes that will be
incorporated into the iRODS dataset.  CDNs would let individual iRODS
deployments scale up the number of reads they could service while preserving the
policies encoded in its rule sets and microservices.  Commodity cloud
storage would allow users to host the results of their computations and share
them with their lab mates and peers before generating and preserving a
``curated copy'' of the data back to iRODS.

\subsection{Challenges}

Augmenting existing systems like iRODS
with commodity infrastructure introduces challenges of
its own.  It is not enough to simply place a CDN in between iRODS and remote readers for
three reasons:

\begin{itemize}
\item \textbf{Protocol Incompatibility}.  CDNs are designed for Web content
acceleration, which means
using HTTP as the data delivery protocol.  However,
iRODS does not speak HTTP.  A protocol translation layer is required.
\item \textbf{Cache Thrashing}.  CDNs are designed for caching lots of
``small'' files---i.e. website assets like HTML or CSS that are not usually
gigabytes in size.  However, iRODS data can be extremely large, and clients may
only even want a small range of an iRODS file.  Serving iRODS data with a CDN
while getting good bandwidth will require file-level fragmentation and reassembly
on both the producer's and consumers' endpoints.
\item \textbf{Cache Incoherency}.  iRODS is a read/write datastore.
While some users are reading from a file, another user can be writing to it.
This can cause readers to cache corrupt data, which in turn gets served to
future readers by the CDN.  Avoiding this problem requires manual coordination
between readers, writers, and the cache operator.
\end{itemize}

In addition, sharing the results of local computations and generating a
dataset to write back to iRODS has its own challenges:

\begin{itemize}
\item \textbf{Replica discovery}.  Suppose a scientist reads some data from iRODS,
runs some local jobs on the data, and saves the job's results to the lab's shared Dropbox
folder.  How do the scientist's peers find the data, so they can run their own
analysis on it?  Today, they email the links to the peers or put the links on the lab
website.  However, this introduces a manual, tedious process for sharing data.  Can we
automate data discovery with Syndicate?
\item \textbf{Replica write-back}.  A scientist's collaborators do not always
have access to her lab's iRODS deployment.  How do her collaborators get their results incorporated
into her deployment?  More specifically, how do they discover a set of
authentication credentials to use to do so?  How does the data ingress server authenticate the
collaborator if they do not have an iRODS account?  Today, the solution is to find
and email an iRODS user with sufficient privileges and ask them to incorporate
the changes.  But can this be done automatically, without requiring users
in the loop?
\end{itemize}

As will be shown, these problems can be solved with the right configuration of Syndicate gateways.

\subsection{Role of SDS}

The need for software-defined storage in scientific computing is not new.  The
labs that gather and share scientific data must already do so according to
data-specific rules.  These include rules governing storage aspects like
national export controls, disclosure of proprietary or potentially dangerous information, and
even mundane concerns like ensuring the data appears in the correct format.

Prior to systems like iRODS, these rules had to be enforced either within the
scientific computing applications, or within a bespoke storage system.
Enforcing the same rules across many labs' applications poses a high cost of
coordination, since each lab's applications must be audited for compliance.
Enforcing a set of rules within a bespoke storage system requires constructing a
bespoke storage system for each rule set.  Allowing a storage system to have its
curation rules programmed at runtime without changing the application-facing
storage APIs is the ``sweet spot'' of SDS for scientific computing.

This Syndicate-powered scientific data-sharing framework extends an existing system (iRODS) with
Syndicate to allow existing workflows to take advantage of commodity infrastructure
(CDNs and cloud storage) without affecting the application-facing storage APIs.
Crucially, the data-sharing framework does so in a way that \emph{preserves} the data owner's existing iRODS
rules in a global setting, while allowing the owner to specify additional
rules within Syndicate to specifically control how data is disseminated once it
leaves iRODS.

\subsection{Design}

An iRODS system can store many different datasets, and each dataset can have its
own access control policies set by the owner.  These are enforced internally by
iRODS when other users attempt to access the data.

\begin{figure}[h]
   \centering
   \includegraphics[width=0.9\textwidth,page=25]{figures/dissertation-figures}
   \caption{Overview of CDN-accelerated scientific data.  The iRODS deployment
   is private, and accessible only via the AG and RG (which run in trusted
   networks).  Remote UGs leverage the MS and CDN to read cached but fresh data,
   regardless of the CDN's caching policies.  When UGs write data, they do so
   via the trusted RG which sends the changes to the proper datasets.  All the
   while, the AG keeps the MS metadata consistent with writes from non-Syndicate
   iRODS clients by subscribing to a (iRODS-specific) message queue.}
   \label{fig:chap4-syndicate-datasets}
\end{figure}

The strategy for distributing each dataset to remote readers is to allow an
iRODS user to ``export'' their dataset by way of a Syndicate AG, and later
``import'' changes to it by way of an RG.  Both the AG and RG run with the
permissions of the dataset owner (Figure~\ref{fig:chap4-syndicate-datasets}).

The AG crawls the owner's dataset using
an iRODS driver and exports individual file metadata to the Syndicate MS.
It runs within a demilitarized zone (DMZ) on the network, linking the iRODS data
to the outside world.  It acts as an origin
server to the CDN and uses its iRDOS driver to load and serve file data as blocks and
manifests to downstream readers.  A dataset owner can run many AGs, and can have
different AGs index different parts of the dataset.

The RG accepts inbound write requests from external users who want to
incorporate their changes to the dataset owner's files.  It also runs in the
DMZ, so it can receive inbound requests.  It uses the dataset owner's
credentials to access iRODS.

The dataset owner's AG and RG operate in \emph{separate} volumes---one for
distributing data to readers (backed by the AG), and one for accepting new data
from writers (backed by the RG).  A remote user would mount the AG-backed volume
for read-only access, and would mount the RG-backed volume for read/write
access.  The AG-backed volume is meant for sharing datasets with collaborators in
wide-area settings using commodity CDNs, while the RG-backed volume is meant for
granting privileged users the ability to save data back into the dataset.
While it is possible for a user to mount
both datasets on the same host, in practice they mount one or the other.

\subsubsection{Read Authorization}

Each AG acts as an origin server to one or more downstream CDNs.  It does not
have direct contact with UGs at the edge, which initiate access flows.  This is
acceptable for public datasets, where no read authentication is needed.

If read confidentiality is desired, the AG will encrypt its
manifests and blocks with the \texttt{serialize()} stage of its driver.  It uses each
UG's public key from the certificate graph to send it a shared secret.  The
encryption is deterministic, such that two requests for the same chunk will
resolve to the same ciphertext.  This allows multiple UGs to leverage the CDN
for read availability without the CDN being trusted with data
confidentiality---UGs reading the same chunk will fetch the same ciphertext, and
the CDN will only cache one copy of the chunk's ciphertext.

The two drawbacks of this configuration is that the CDN can still see access patterns
on the ciphertext, and anyone who can read from the CDN can fetch ciphertext.
This may allow unauthorized principals to infer information about the data being
accessed.  If scientists wish to avoid this outcome, then the solution is to use
a trusted CDN that will carry out authentication at the edge.

Regardless of the disposition of the CDN, the scientific applications are none
the wiser as to the authentication steps taken by the UG.  This is
because the UG's driver handles the interfacing with the CDN.  If the
UG needs a decryption key to read chunks (i.e. for read confidentiality), then
the volume owner can distribute them to the UGs by sharing it through the
certificate graph (encrypting the decryption key with each UG's public key).
The UG fetches and decrypts the key automatically, as part of its driver code.  If
the UG needs an access credential to the CDN (i.e. for metadata
confidentiality), then the volume owner can use the certificate graph to encrypt
and distribute it to remote UGs and write the CDN driver to submit the
credential to the CDN on read.

\subsubsection{Write Authorization}

Each RG acts as an aggregation point for data generated by external UGs.
Unlike the read path, the UGs have direct contact with the RG on the write path.
As such, the UG's Push stage can ensure data confidentiality across the wide
area simply by contacting the RG via a TLS channel using client-side certificates.  The UG and RG configurations in the
certificate graph would be structured to include TLS keys and certificates for each
gateway, so the RG could authenticate UGs and UGs could authenticate the RG.

Only the RG has write access to the iRODS deployment.  The volume owner installs its
iRODS credentials via the certificate graph as well, ensuring that they are
confidential and up-to-date by encrypting them with the RG's public key.

The RG has the ability to perform write authorizations on a per-file and even a
per-file-region basis.  This is because the UG informs the RG which file it
writes to (i.e. as part of the manifest and block information it Pushes), and it
informs the RG whenever it renames or truncates a file.  The RG has the ability
to NACK operations that do not conform to the volume owner's policies.  For
example, a UG may be denied a request to rename a file into a separate
\texttt{\$HOME} directory in order to prevent users from gaining control of
parts of the dataset.

\subsubsection{Preserving Existing Policies}

From iRODS's perspective, the AG and RG are the only readers and writers to a
particular dataset.  Moreover, their reads and writes reflect the global
sequences of reads and writes initiated by wide-area users.  As such, the
volume owner retains the power to \emph{globally} enforce her iRODS-specific rules on data
accesses---any gateway-initiated accesses must also conform to the
already-deployed iRODS rules.

The volume owner can also set \emph{per-user} rules within her
gateways' drivers.  Importantly, iRODS does not need to be aware of the
Syndicate users, nor aware of the per-user policies that the gateways enforce,
since they occur in a separate layer.

As a result, scientists do not need to do any extra work or carry out any extra
configuration steps to begin sharing their iRODS-hosted data with off-site users.
All of their iRODS-local access policies continue to apply, and the scientist
has the \emph{option} of creating more-detailed rules within Syndicate.  The
only step the scientists must take is starting up a publicly-accessible AG and
RG, which will read and write to their datasets on their behalf when remote
users request it.

\subsection{Implementation}

The prototype data-sharing framework employs a variant of the
Akamai~\cite{akamai} CDN deployed on
OpenCloud~\cite{opencloud}---a federated computing platform similar to PlanetLab.
The CDN is operated by the OpenCloud developers and is made available to all
participating sites.

Several AGs have been deployed on top of the iRODS deployment at the University of
Arizona, and registered as origin servers on the OpenCloud-hosted CDN.  In
addition, several Docker images~\cite{syndicate-website} and a dataset-mounting
tool~\cite{sdm} have been made 
available for the general public to try out the system.

When a user downloads and installs a read/write Docker image, she receives two
mounted volumes---the read-only volume containing the dataset, and a read/write
volume for writing the results of her experiment.  She and her collaborators
will see each other's results when they are written.  Several RGs are deployed
that will write back her and other users' results both to iRODS and to a temporary S3
bucket that gets cleared every 24 hours.

iRODS-compatible applications interface with iRODS through FUSE and through a
set of command-line utilities.  The framework comes with a FUSE filesystem to preserve
compatibility.

A Hadoop filesystem plugin has been implemented that allows Hadoop
computing jobs to pull data from iRODS via Syndicate and the underlying CDN.
The HDFS plugin gives the job scheduler insight as to where Syndicate UGs have
locally cached chunks, so it can schedule tasks on nodes that already have the
data present.

The iRODS-facing driver is ~1900 lines of Python, of which ~640 are specific to
iRODS, ~375 are specific to the AG, and ~375 are specific to the RG.  The
UG-specific CDN interfacing driver is ~100 lines of Python.  The Hadoop plugin
is ~2300 lines of Java.

\subsection{Discussion}

Again, the utility of using SDS to link existing scientific data stores to
commodity cloud infrastructure is that SDS allows users to ``slap
on'' extra storage and data distribution capacity with little effort.  The
marginal cost of adding support for new storage and CDNs in terms of lines of
code is small enough that it can be achieved in as little time as a couple
of hours, including testing.

The key benefit to iRODS users in particular is that the iRODS system needs no
special modification to be made compatible with the CDN.  This is because
SDS effectively ports the CDN and storage to iRODS, instead of the other way around.
In doing so, iRODS-compatible applications (and even applications that only use
iRODS indirectly, such as Hadoop jobs) can transparently benefit from its extra
availability without having to overcome these aforementioned challenges.

\section{Remarks}

In these sample applications, using an SDS system to host data addresses
several difficult problems.  In all three examples, 
\emph{SDS provides semantic independence from storage systems} (i.e.
portability).  By both wrapping individual services behind well-defined service driver models, and by
allowing the volume owner to inject aggregation driver logic in-between the
application endpoints and the underlying services, the SDS systems ensured that applications
had access to a persistent data store that behaved exactly they way they needed
\emph{as if the application was using a purpose-built storage system}.

In all cases, the desired application-specific storage semantics were realized
by implementing gateways to handle unrelated storage concerns, and composing them
together into handle access and mutate flows.
In the Gaia groupware-powered applications, users host their data on whatever storage providers they want
while deploying gateways to enforce arbitrary access controls and maintain globally-visible search
indexes.  In the email example, users host
their email in whatever storage providers they want and deploy gateways to
implement spam filtering, search indexes, message prioritization.  In the scientific
data-sharing example, users augment existing storage systems with the CDNs and cloud
storage of their choice while deploying gateways to preserve the original system's global access
control rules and end-to-end consistency semantics.  Being able to compose
gateways was crucial to preserving storage semantics across organizations, since
in all cases there were sensitive operations like access controls and encryption
that could only be allowed to execute on certain computers.

The separation between aggregation drivers and service drivers proved useful in
practice.  Having these two layers of indirection allows service
drivers to be reused across different applications, and even across different SDS systems.
In doing so, \emph{SDS reduced the marginal cost of adding support for new
services to writing a single driver}.
Neither the aggregation driver logic nor the application need to be
patched when a new service becomes available.  Instead, a volume owner only
needs to deploy an updated service driver to her relevant gateways.

A slew of engineering problems were solved by
designing the storage layers of these applications to treat users as the
de-facto data owners.
For groupware, this enabled gateways to authenticate and vet new users who would
share their groupware data.  For email, this enabled gateways to discover the
sender and recipient public keys automatically and preserve end-to-end
authenticity and confidentiality.  For scientific data sharing,
this enabled gateways to identify and preserve dataset access controls regardless of the infrastructure used
to ship the data to external consumers.


\chapter{Evaluation}
\label{chap:evaluation}

This chapter presents performance numbers for Gaia and Syndicate.
Ultimately, the end-to-end performance of an SDS system depends on the decisions
made in the application-specific aggregation and service driver implementations.
 However, each SDS system will impose measureable, predictable overheads on
reads and writes.  It is important for developers to know where these overheads
come from and how big they are in order to make good driver and application design decisions.
In light of the measured overheads, this chapter gives developers
recommendations on how to implement their aggregation drivers to minimize the
impact for specific workloads and end-to-end semantics.

\section{Overview}

SDS systems offer developers a trade-off.  On the one hand, the cost to
developers and users is some additional performance overhead on the read and write paths to cloud services.
This is because the data needs to be converted
into access and mutate flows comprised of manifests and chunks, which must be
relayed through one or more gateways en route to the cloud services.
The SDS metadata service may also need to be contacted in order to
complete the operation.

On the other hand, SDS systems offer significant gains over the status quo.
First, SDS systems greatly reduce the cost of building and maintaining
system-of-systems applications built on cloud services.  By handling storage
semantics and inter-organizational trust relationships
independent of applications, SDS systems allow developers to spend more time on
their application business logic and less time on addressing the users'
storage and trust concerns.  Second, SDS applications
give organizations unilateral control over how
their data is hosted, which frees developers from having to be
responsible for conforming to their hosting policies.
Third, SDS decouples application data from the application
code, preventing users from being locked into relying on a specific
application.  These gains are realized in each
sample application presented in Chapter~\ref{chap:applications}.

Despite overheads, using a SDS system does \emph{not} mean that application workloads take more time or
space than they would had the application avoided SDS.  In fact,
the workload on the SDS system can be \emph{faster} and \emph{require less
space}, depending on the behavior of the service and aggregation drivers.
For example, a service driver can cache blocks on behalf of a slow service like Amazon Glacier~\cite{amazon-glacier},
allowing a read-heavy workload to execute faster with the SDS system
than it would have if the workload had to contact the service directly.
As another example, an aggregation driver can deduplicate
and compress chunks before sending them to service drivers, speeding up data
transmission and reducing the amount of storage space needed when compared to
directly writing to storage services.

This chapter describes and measures the time and space overheads in both
Syndicate and Gaia's \emph{default} access and mutate flow behaviors, and
provides recommendations on how to design an aggregation driver to minimize
their impact on common workloads.  The measurement focus is on the
\emph{efficiency} of reads and writes---that is, what fraction of time and space
is used for loading and storing the application's data.
Some aggregation driver design recommendations distilled from
these measurements have already been used in real-world applications built
on these SDS systems.

The efficiency measurement gives developers a way to
measure the impact of their service and aggregation driver implementations on
the end-to-end overhead perceived by the application.
A higher efficiency implies higher application-perceived goodput.  Performant
driver implementations maximize efficiency using tactics such as caching
MS-obtained data, or running access and mutate flow steps in parallel when
possible.

To maximize efficiency, the developers need to first understand the SDS system's
overheads in order to make good engineering decisions to minimize their
impact on reads and writes.  But in order to write an effective aggregation
driver, developers need to know what overheads exist in a SDS system first.

\section{Access Flows}

An access flow runs in two steps:  the Discover step translates the
name of a record into a manifest ID, and an Acquire step uses the manifest
ID to fetch the block(s) that contain the requested data.  Both steps may be
implemented in the aggregation driver, but the SDS system supplies a default
implementation if the aggregation driver does not.
This section presents the time and space overheads of the systems' default
access flow behavior, so developers can better understand how to design
aggregation drivers and applications to minimize their effect on their
workflows.

\subsection{Overheads in Gaia}

In both Gaia and Syndicate, the default behavior of the Discover step is to query the
metadata service.  This adds measurable time overhead to the access flow's
execution, comprised of a network round-trip plus the time required by the MS
implementation to look up the current manifest ID for the record.

Gaia is designed for applications where each user is the owner of a volume that
contains all of their application-specific data.
When using a Gaia-powered application, the volume owner usually
has only one device online---the one she is
currently using.  To take advantage of this, Gaia provisions gateways to run 
both the Discover and Acquire step on the volume owner's personal device by
default.  The node itself participates in the peer-to-peer metadata service,
and in doing so maintains a local copy
of all of the users' volume certificate graphs and pointers to Gaia metadata.
In addition, the node maintains a set of service drivers that its gateways can use to load and store
chunks on behalf of the organization that runs it.

The default Discover step in Gaia is to try to fetch the volume record from each
storage system it has a driver for.  This includes
looking up the volume owner's public key and 
certificate graph in the node's local replica of
the system's set of zone files, since this information is required to
authenticate the record.  The volume record itself is stored in the volume
owner's chosen cloud storage services, meaning that the gateway running the
Discover step only has to ask its
node's colocated storage driver to fetch it and decode it.  In the worst case,
the Discover step has to try each storage service before succeeding.

The volume record contains the volume's manifest ID (recall that a Gaia volume
only has a single manifest).  This is passed to the Acquire step, which by
default will fetch each of the volume owner's key space shards in parallel and
reassemble them into the list of keys available in the volume.  Once the full
set of keys are assembled, the Acquire step can finally fetch the requested
value (as a single chunk).

To fetch the value's chunk, the default Acquire step looks at the volume record
to get a list of upstream Gaia gateways or storage providers that the volume
owner has listed as possibly storing a copy.  The Acquire step iterates through
them in order, using one of the node's service drivers to contact each service.
Functionally speaking, the act of asking an upstream Gaia gateway for the chunk is 
equivalent to asking a storage provider---to the requesting
gateway, there is no difference in mechanism (i.e. the upstream gateway is
treated like a storage provider).

To summarize, the sources of overhead in Gaia's default access flows stages are:

\begin{itemize}
\item \textbf{Fetching the volume record in the Discover step}.  This adds both time overhead in
the form of a round-trip to a storage service, and a constant-space overhead from having
to store the volume record (which contains the manifest IDs).
\item \textbf{Fetching the manifest in the Acquire step}.  This
is the work required to find the globally-consistent view of the manifest.
Each device that can write to a Gaia volume stores its own manifest, so a reader
will need to fetch all of them and merge them.  This adds both a time
overhead in the form of a round-trip to a storage service (one per device
manifest), and a $O(kn)$ space overhead for $n$ keys and $k$ devices.  The time overhead of
interest is the combined time
required to fetch all device manifests in parallel, and the time required to authenticate
each device's signature.  The space
overhead is dominated by $n$, since $k$ is the number of devices the volume
owner can use to write (which in practice is small---on the order of 10 or fewer).
Each key requires a constant amount of space.
\item \textbf{Decoding and authenticating the chunk fetched from the Acquire
step}. This adds a time overhead that is $O(n)$ in the number of bytes, since the chunk must
be hashed.
\end{itemize}

\subsubsection{Measured Overheads}

Gaia's read performance was evaluated for records whose sizes vary between 64K
and 640K, in intervals of 64K.  These sizes
were chosen to emulate the sizes of files being stored.  For example, a typical
image would be around 640k (taken from Blockstagram~\cite{blockstagram}),
a typical text document would be around 320k (taken from Graphite
Docs~\cite{graphite-docs}), and the set of the user's
microblog posts would be around 64k (taken from Publik~\cite{publik}).

The read tests were done on a Gaia deployment that matches the \emph{cold start}
default deployment given to each user.
In this deployment, the user runs a local Gaia node that Discovers and Acquires data
from an upstream Gaia node.  The upstream Gaia node loads and stores data from
Microsoft Azure Blob Storage~\cite{azure-blob-storage} in response to downstream
Gaia requests.  Anyone can read from it, but it only accepts writes from users
who have stored profiles that point to at least two social media accounts (i.e.
in order to prevent bots from abusing the system).  In this test, no caching of
any kind was performed by the local Gaia node.

This test evaluates a production deployment of Gaia that serves over
15,000 users.  It is used by the majority of users of the Gaia-powered
applications described in Chapter~\ref{chap:applications}.

An instrumented version of a Gaia node was used to evaluate 150 reads on these
file sizes on the live Gaia network.  A single record was written to an empty
volume and read back 150 times from a single device.
The overall read performance is shown in
Figure~\ref{fig:gaia-read-total}.

\begin{figure}[htp!]
   \centering
   \includegraphics[width=\textwidth]{figures/results/gaia-reads.all/Total}
   \caption{Box-and-whiskers plot of the overall read performance.}
   \label{fig:gaia-read-total}
\end{figure}

Unsurprisingly, the time taken to store data increases linearly with the amount
of data being stored.  Storing a ``small'' file of 64K took a median of 0.836
seconds on the cold path, and an average of 1.56 seconds with a standard
deviation of 1.79.

Each read operation is composed of a Discover and Acquire stage.  The
performance of both of these stages is shown in
Figure~\ref{fig:gaia-read-stages}.

\begin{figure}[htp!]
   \centering
   \begin{subfigure}[b]{.8\textwidth}
      \includegraphics[width=\textwidth]{figures/results/gaia-reads.all/Discover}
      \label{fig:gaia-read-discover}
      \caption{Gaia Discover performance.}
   \end{subfigure}
   \begin{subfigure}[b]{.8\textwidth}
      \includegraphics[width=\textwidth]{figures/results/gaia-reads.all/Acquire}
      \label{fig:gaia-read-acquire}
      \caption{Gaia Acquire performance.}
   \end{subfigure}
   \caption{Box-and-whiskers plots of access flow stage performances in Gaia,
   for file sizes between 64K and 640K in increments of 64K.}
   \label{fig:gaia-read-stages}
\end{figure}

As expected, the Discover stage time overhead is constant relative to the file
size---all that is happening in this stage is the constant-sized 
volume record is being loaded.
Also, as expected, the Acquire stage time increases with file size, since
it includes the times taken to fetch the (single) manifest and fetch the (single) block.
The Acquire stage time overheads are further broken down in
Figure~\ref{fig:gaia-acquire-breakdown}.

\begin{figure}[htp!]
   \centering
   \begin{subfigure}[b]{.8\textwidth}
      \includegraphics[width=\textwidth]{figures/results/gaia-reads.all/GetManifest}
      \label{fig:gaia-read-getmanifest}
      \caption{Times taken to fetch the manifest in the Acquire step, for given
      key/value block sizes.}
   \end{subfigure}
   \begin{subfigure}[b]{.8\textwidth}
      \includegraphics[width=\textwidth]{figures/results/gaia-reads.all/GetBlocks}
      \label{fig:gaia-read-discover}
      \caption{Times taken to fetch the key/value block in the Acquire step, for
      given sizes.}
   \end{subfigure}
   \caption{Box-and-wiskers plots of the Acquire stage performance.}
   \label{fig:gaia-acquire-breakdown}.
\end{figure}

In this test, the manifest size was constant, and took between 0.522 and 0.576 
seconds to download in the median case.  The times taken to download the record
increased with the record size, taking between 0.223 seconds (for 64K) and 0.448
seconds (for 640K) in the median cases.

The efficiency for each read was calculated for record sizes tested.  Two
efficiency results were calculated.  The ``cold efficiency'' is calculated as
the time taken to fetch the record data
divided by the total running time of the access flow.  This includes fetching
the volume record from the underlying cloud storage provider, despite the fact
that in practice this record can be safely cached.  The ``warm efficiency'' is
similar to the ``cold efficiency'', except the total running time of the access
flow \emph{excludes} the time taken to fetch the volume record.  This
calculation reflects the act of caching the volume record.

\begin{figure}[htp!]
   \centering
   \begin{subfigure}[b]{.8\textwidth}
      \includegraphics[width=\textwidth]{figures/results/gaia-reads.all/Efficiency}
      \label{fig:gaia-read-getmanifest}
      \caption{The cold efficiency, which includes the time taken to fetch the
      volume record over the network.}
   \end{subfigure}
   \begin{subfigure}[b]{.8\textwidth}
      \includegraphics[width=\textwidth]{figures/results/gaia-reads.all/Efficiency_cache}
      \label{fig:gaia-read-discover}
      \caption{The warm efficiency, which assumes the volume record is cached
      and excludes it from the access flow's running time}
   \end{subfigure}
   \caption{Box-and-wiskers plots of Gaia's read efficiencies.}
   \label{fig:gaia-read-efficiencies}.
\end{figure}

Both efficiencies are shown in Figure~\ref{fig:gaia-read-efficiencies}.  Naturally,
the efficiencies will tend towards 1.0 for larger and larger record sizes.
However, the act of caching the volume record can significantly improve the read
efficiency for small records.

\subsubsection{Recommendations for Developers}

Relative to fetching data directly from a storage provider, Gaia's most
noticeable time overheads come from fetching and decoding all of the metadata
and volume record information required to request the data.
Fortunately, these costs can be
mitigated by metadata-reading and metadata-writing strategies in
their applications' aggregation drivers.

\hfill \break
\noindent{\textbf{Metadata Caching}}
\hfill \break

In applications that assume at most one writer per login session,
the Gaia gateways can cache data for the duration of the session.
For example, the Gaia-powered Coins application~\cite{coins} assumes that the
user only accesses the data from the same trusted device.  This application's
aggregation driver can
spare users the cost of reloading metadata on each read by having its Acquire
and Publish steps maintain a coherent local copy.

Some applications only need delta-consistency~\cite{delta-consistency}.  In such
applications, users are already expected to have to wait a few seconds for newly-written data to
appear.  In this case, the Acquire step can be implemented to cache metadata for
a particular key for an application-configurable amount of time.  For example, a
social media application could cache the metadata for a user's avatar for a long
time, since it is not expected that the user will change it frequently.  This
would save other readers the cost of having to fetch the same metadata for it 
over and over.

The exact rules for how long to cache a given key/value pair's metadata are
application-specific, and affect the end-to-end storage semantics of the volume.
As such, metadata caching is not part of Gaia's default
behavior.  Instead, Gaia defers to the aggregation driver to make these
decisions.

In practice, the volume record is cached for the duration of the session.  This
was omitted from the read test since the test was meant to
measure overheads from cold starts (in order to identify overhead points).
Simply caching this record for the duration
of a login shaves about 0.16 seconds off of a read.

\hfill \break
\noindent{\textbf{Metadata Streaming}}
\hfill \break

Some applications like online shared document editors need to access metadata quickly
from a small set of peers.  In these cases, the Acquire and Publish steps of the
aggregation driver can augment the default read path by eagerly replicating
their metadata via a shared broadcast channel (such as a shared Web socket), in
addition to replicating it to cloud storage.
This way, the Acquire step would listen for Publish events from other peers, and
eagerly update cached metadata before an application read occurs.  If the
broadcast channel is down or starts dropping messages, the Acquire step would fall back
to fetching metadata via the slow path.

Discovering the broadcast channel can be achieved by placing hints in the
volume record in the Gaia MS, such as a set of Websocket URLs.  On loading, the
application would allow the user to connect to other peers by querying the
peers' volume records to find their broadcast channels of choice.

Not all applications need this feature, and even for applications that do need
it, the developer would need to select a broadcast channel that is capable of
handling the application's workload.  As such, Gaia defers
metadata streaming to the aggregation driver implementation.

At least two Gaia-powered applications---Stealty~\cite{stealthy.im} and
Hermes~\cite{hi-hermes}---are known to use this technique.  Both are
encrypted chat applications, and use a streaming mechanism to accelerate message delivery instead of
forcing clients to continuously poll each others' Gaia volumes.  They
both replicate message logs to Gaia so recipients can read messages
sent to them while they were away.

\subsection{Overheads in Syndicate}

In Syndicate, the Discover and Acquire steps always run on user gateways.
In the default case, the same user gateway runs both
an access flow's Discover and Acquire steps.

The default behavior of Syndicate's Discover step is to contact the
cloud-hosted MS for the latest manifest ID.  Unlike Gaia, Syndicate's
metadata records are arranged into a filesystem-like file hierarchy, and
gateways do not need to obtain a full record of the volume metadata in order to access
blocks.

The time overhead of fetching a given metadata record in Syndicate 
is dependent in how deep into the metadata
hierarchy it is, since at a minimum the user gateway will need to verify that
its cached copies of the path's directory logs are up-to-date.
If the directory's log is not cached or has been modified
since the last request, then the overhead will also include fetching and
synchronizing each directory log in the path.
The space overhead for processing the metadata path 
is simply the sum of the metadata record fetched, plus the
sum of the sizes of the directory logs.

The default behavior of the Acquire step in Syndicate is to first fetch the
manifest from an upstream gateway, and then fetch the relevant blocks from one
or more upstream gateways.  It uses the metadata record from the Discover step
to determine whether or not it needs to contact an acquisition gateway or the
volume's replica gateways.

When reading a record whose data is hosted in a curated dataset, the user
gateway's default Acquire step always contacts the acquisition gateway that
acts as its coordinator.  It determines which gateway this is using the metadata
record acquired in the Discover step.  The acquisition gateway is not given any
any specialized aggregation driver logic by default.  It responds to the user
gateway by using its service
driver to fetch and serve the requested block or manifest from the underlying
data set.

When reading a record whose data is stored in a cloud storage provider, the user
gateway may choose from a set of replica gateways that may be able to access it.
The default behavior of the Acquire step in this case is to ask each replica
gateway in order of increasing gateway ID.  Once the user gateway successfully
fetches the manifest, it fetches at most six blocks in parallel from the replica
gateways.  Similar to how it fetches the manifest, the user gateway will try
contacting each replica gateway in sequence by replica ID for the block (but
will process at most six blocks concurrently).
The choice of six parallel connections is inspired by the same implementation choice made in Web
browsers~\cite{browserscope-browsertest}.  % http://www.browserscope.org/?category=network
The replica gateways are not given any aggregation driver logic for the Acquire
step by default; they simply use their service drivers to fetch and serve chunks
from their services upon request.

To summarize, the overheads in Syndicate's default access flow stages are:

\begin{itemize}
\item \textbf{Synchronizing the metadata path's directory logs}.
In the best case, all of the path entries will be cached and up-to-date.  In
this case, the time overhead is $O(n)$ for $n$ path entries, and $O(dn)$ space
overhead for $d$ entries per directory log.  In the worst case,
all path entries will not be cached.  In this case, both the time and space
overheads are $O(dn)$.
\item \textbf{Fetching blocks and manifests through a replica or acquisition gateway}.  When
servicing an application read, the user gateway does not fetch the data directly
from the cloud service or dataset, but instead contacts an upstream replica
gateway or acquisition gateway to
fetch it on its behalf.  However, this overhead is only
incurred when the user gateway cannot load the requested block or manifest from
an upstream CDN.
\item \textbf{Fetching, authenticating, and decoding the manifest}.
This adds $O(m)$ time and space overhead in the best case (i.e. the first
replica gateway contact has the manifest),
where $m$ is the number of blocks in the record.  In the worst case, the time overhead is
$O(m+r)$ for $r$ replica gateways, since in the worst case the last replica
gateway to be contacted out of the set of replica gateways in the volume
has the manifest.
\item \textbf{Searching for the correct replica gateway}.  Fetching a manifest
or block that is available only from replica gateways incurs at worst a $O(g)$
overhead, for $g$ replica gateways.  This is due to the simple but inefficient
strategy of contacting replica gateways in order by gateway ID.  The $g$
parameter is not expected to change very often relative to the occurence of reads and
writes, since the user is not expected to frequently add and remove gateways.
\item \textbf{Fetching and decoding the data as a set of blocks}.  A record that exceeds the
volume block size will be fetched piecemeal over HTTP.  This adds $O(n/b)$ time
and space overhead, where $n$ is the number of bytes and $b$ is the block size.
The overheads come from processing and discarding the HTTP headers.  In
addition, each block will be hashed in order to authenticate it,
yielding a $O(n/b)$ time overhead.
\end{itemize}

\subsubsection{Measured Overheads}

Syndicate reads and writes data as fixed-sized blocks.  Obviously, a larger
block size and a larger file size would improve the efficiency of Syndicate
reads and writes, since a greater fraction of the total operation time would be
spent uploading or downloading data.  Therefore, read performance was measured
on a small block size of only 1k, and on a moderate block size of 10k, in order
to emphasize read overheads in the measurements.

File sizes were chosen between 10x and 100x the block size, in intervals of 10
blocks (meaning 10 different file sizes were tested per block size).  Each file
of each size was downloaded 100 times to measure overheads.
The read test was carried out on a volume with one UG and one RG.
The RG and MS ran on a Microsoft
Azure VM and stored metadata and chunks to the VM's local disk.

The Discover step in these tests synchronized one directory (the root) and fetched
one metadata record.  The Acquire step in these tests fetched a manifest representing between 10 and
100 blocks.  The size of the block's metadata in the manifest is
constant---manifest sizes and download times are a function of the number of
blocks only.  However, the size of the record metadata on the MS increases with
the number of blocks, since the MS stores a garbage-collection log for all
blocks in the record.  Each garbage-collection entry is only 8 bytes.


\begin{figure}[htp!]
   \centering
   \begin{subfigure}[b]{.8\textwidth}
      \includegraphics[width=\textwidth]{figures/results/syndicate-writes.1024.all/Total}
      \label{fig:syndicate-read-total-1k}
      \caption{Syndicate access flow performance with 1K blocks}
   \end{subfigure}
   \begin{subfigure}[b]{.8\textwidth}
      \includegraphics[width=\textwidth]{figures/results/syndicate-writes.10240.all/Total}
      \label{fig:syndicate-read-total-1k}
      \caption{Syndicate access flow performance with 10K blocks}
   \end{subfigure}
   \caption{Box-and-whiskers plots of end-to-end access flow performances in
   Syndicate, for 1K and 10K block sizes.}
   \label{fig:syndicate-read-total}
\end{figure}

The total read performances for 1K and 10K block sizes are shown in
Figure~\ref{fig:syndicate-read-total}.  These represent the times taken by all
of the Discover and Acquire logic, including the aforementioned overheads.

\begin{figure}[htp!]
   \centering
   \begin{subfigure}[b]{.8\textwidth}
      \includegraphics[width=\textwidth]{figures/results/syndicate-reads.1024.all/Discover}
      \label{fig:syndicate-read-discover-1k}
      \caption{Syndicate Discover performance with 1K blocks}
   \end{subfigure}
   \begin{subfigure}[b]{.8\textwidth}
      \includegraphics[width=\textwidth]{figures/results/syndicate-reads.1024.all/Acquire}
      \label{fig:syndicate-read-acquire-1k}
      \caption{Syndicate Acquire performance with 1K blocks}
   \end{subfigure}
   \caption{Box-and-whiskers plots of access flow stage performances in
   Syndicate, for file sizes between 10K and 100K and a block size of 1K.}
   \label{fig:syndicate-read-stages-1K}
\end{figure}

\begin{figure}[htp!]
   \centering
   \begin{subfigure}[b]{.8\textwidth}
      \includegraphics[width=\textwidth]{figures/results/syndicate-reads.10240.all/Discover}
      \label{fig:syndicate-read-discover-10k}
      \caption{Syndicate Discover performance with 10K blocks}
   \end{subfigure}
   \begin{subfigure}[b]{.8\textwidth}
      \includegraphics[width=\textwidth]{figures/results/syndicate-reads.10240.all/Acquire}
      \label{fig:syndicate-read-acquire-1k}
      \caption{Syndicate Acquire performance with 10K blocks}
   \end{subfigure}
   \caption{Box-and-whiskers plots of access flow stage performances in
   Syndicate, for file sizes between 100K and 1000K and a block size of 10K.}
   \label{fig:syndicate-read-stages-10K}
\end{figure}


Figures~\ref{fig:syndicate-read-stages-1K} and \ref{fig:syndicate-read-stages-10K}
show the Discover and Acquire stage performances of Syndicate access flows for 1K and
10K blocks, respectively.
In both cases, the times taken by the Discover step increase slightly for
records with 80, 90, and 100 blocks.  This is due to the fact that the MS
incurs extra disk overhead from loading the a metadata record with a larger
garbage-collection log.  This could be optimized away.

Unsurprisingly, the Acquire stages increase in time as a linear function of the
number of blocks fetched.  The spreads of the distributions increase
with the number of blocks because more blocks introduce more noise into the
measurement.  The time taken to fetch records increases faster for 10K blocks
than for 1K blocks.


\begin{figure}[htp!]
   \centering
   \begin{subfigure}[b]{.8\textwidth}
      \includegraphics[width=\textwidth]{figures/results/syndicate-reads.1024.all/GetManifest}
      \label{fig:syndicate-getmanifest-1k}
      \caption{Syndicate's performance of fetching manifests with 1K blocks}
   \end{subfigure}
   \begin{subfigure}[b]{.8\textwidth}
      \includegraphics[width=\textwidth]{figures/results/syndicate-reads.1024.all/GetBlocks}
      \label{fig:syndicate-getblocks-1k}
      \caption{Syndicate's performance of fetching records of various sizes with
      1K blocks.}
   \end{subfigure}
   \caption{Box-and-whiskers plots of Syndicate's Acquire stage performance with 1K
   blocks.}
   \label{fig:syndicate-acquire-breakdown-1K}
\end{figure}

\begin{figure}[htp!]
   \centering
   \begin{subfigure}[b]{.8\textwidth}
      \includegraphics[width=\textwidth]{figures/results/syndicate-reads.10240.all/GetManifest}
      \label{fig:syndicate-getmanifest-10k}
      \caption{Syndicate's performance of fetching manifests with 10K blocks}
   \end{subfigure}
   \begin{subfigure}[b]{.8\textwidth}
      \includegraphics[width=\textwidth]{figures/results/syndicate-reads.10240.all/GetBlocks}
      \label{fig:syndicate-read-acquire-1k}
      \caption{Syndicate's performance of fetching the blocks for records of various sizes with
      10K blocks.}
   \end{subfigure}
   \caption{Box-and-whiskers plots of Syndicate's Acquire stage performance with
   10K blocks.}
   \label{fig:syndicate-acquire-breakdown-10K}
\end{figure}

The tasks of fetching the manifest ID and fetching
the blocks for the Acquire stages are further broken down in
Figures~\ref{fig:syndicate-acquire-breakdown-1K} and
\ref{fig:syndicate-acquire-breakdown-10K}.

The times taken to fetch the manifests are about the same across the record
and block sizes sizes measured.  While
the size of a manifest grows linearly with the number of blocks, it does so via
a small constant factor per block (48 bytes per block).  The times taken to
fetch the blocks are the reason why the Acquire stage's time increases linearly
with the record size.

\begin{figure}[htp!]
   \centering
   \begin{subfigure}[b]{.8\textwidth}
      \includegraphics[width=\textwidth]{figures/results/syndicate-reads.1024.all/Efficiency}
      \label{fig:syndicate-cold-efficiency-1k}
      \caption{The cold efficiency for 1K blocks, which includes the time taken to fetch the
      metadata record from the MS.}
   \end{subfigure}
   \begin{subfigure}[b]{.8\textwidth}
      \includegraphics[width=\textwidth]{figures/results/syndicate-reads.1024.all/Efficiency_cache}
      \label{fig:syndicate-warm-efficiency-1k}
      \caption{The warm efficiency, which assumes the metadata record is cached
      and excludes it from the access flow's running time}
   \end{subfigure}
   \caption{Box-and-wiskers plots of Syndicate's read efficiencies for 1K blocks.}
   \label{fig:syndicate-read-efficiencies-1k}
\end{figure}

\begin{figure}[htp!]
   \centering
   \begin{subfigure}[b]{.8\textwidth}
      \includegraphics[width=\textwidth]{figures/results/syndicate-reads.10240.all/Efficiency}
      \label{fig:syndicate-cold-efficiency-10k}
      \caption{The cold efficiency for 1K blocks, which includes the time taken to fetch the
      metadata record from the MS.}
   \end{subfigure}
   \begin{subfigure}[b]{.8\textwidth}
      \includegraphics[width=\textwidth]{figures/results/syndicate-reads.10240.all/Efficiency_cache}
      \label{fig:syndicate-warm-efficiency-10k}
      \caption{The warm efficiency, which assumes the metadata record is cached
      and excludes it from the access flow's running time}
   \end{subfigure}
   \caption{Box-and-wiskers plots of Syndicate's read efficiencies for 10K blocks.}
   \label{fig:syndicate-read-efficiencies-10k}
\end{figure}

Figures~\ref{fig:syndicate-read-efficiencies-1k} and
\ref{fig:syndicate-read-efficiencies-10k} plot the cold and warm efficiencies of
Syndicate, for 1K blocks and 10K blocks respectively.  In both cases, the
efficiencies approach 1.0 as the record sizes increases.  The warm efficiency
excludes the time taken to fetch the metadata record from the MS (i.e. by
caching it locally).  Caching metadata makes a noticeable difference in the
system's efficiency for small files---it can be up to 33% higher.

\subsubsection{Recommendations for Developers}

Syndicate gives developers several options to manage read performance in the
face of these overheads.  A few of these recommendations have been put into
practice in production settings.

\hfill \break
\noindent{\textbf{Long Metadata TTL with Explicit Invalidation}}
\hfill \break

Volumes of scientific data often have few writers.  In cases where a volume is
backed by a dataset, the only writer would be the acquisition gateway that
crawls the dataset.  In cases where a volume acts as a data dump or a scratch space,
writes happen only when a workload finishes, and occur on the same set of
metadata paths (e.g. each user or each workflow would write its dump to its own
directory).

Developers can take advantage of these special cases to save round-trips to the MS.
The Publish steps of writer gateways can be programmed to 
broadcast a metadata invalidation hint to all read-capable gateways in the
volume.  The MS would only be contacted as a fallback.

This strategy is used in the scientific data-sharing application today in order
to improve the efficiency of reading small files.

\hfill \break
\noindent{\textbf{Use a CDN}}
\hfill \break

Syndicate was designed to be used with a CDN.  Developers wishing to get the
best read performance would implement their Acquire step to contact one or more
CDNs that can pull down chunks from upstream replica gateways.  This is highly
beneficial for read-heavy workloads, where most of the chunks can be cached
close to readers.  This reduces the number of network round-trips and reduces
the amount of transit traffic out of cloud storage providers, all without
violating end-to-end storage semantics and organizational autonomy.

The performance boost developers can expect to see depends on the CDN leveraged
and the size of the working set.
However, the benefit to breaking data into chunks is that
developers can expect the CDN to accelerate reads even if only
part of the data is cached.

This strategy is used in the scientific data-sharing application today.  The
CDN---an instance of the Akamai~\cite{akamai} CDN---is deployed on OpenCloud~\cite{opencloud}.

\hfill \break
\noindent{\textbf{Gateway-local Block Cache}}
\hfill \break

Since Syndicate handles end-to-end semantics at a level above block transport,
each gateway can implement a write-coherent block cache in its Acquire stage. 
This effectively adds multiple tiers to a commodity CDN---the first tier would be at the user
gateways, the CDN would be the shared middle tier, and the replica and acquisition
gateways would be the top tier.  Syndicate gateways offer this feature as a
built-in option, but using it requires the developer to set the cache size first
(which is workload-specific).

This strategy is deployed in the scientific data-sharing application.

\hfill \break
\noindent{\textbf{Chunk Advertisement}}
\hfill \break

If the developer implements a gateway-local block cache in the Acquire step, a
complementary feature would be allowing gateways to advertise to one another
which chunks they have cached.  If the Acquire step detects that a nearby peer
has a cached chunk, then it could fetch the chunk from the nearby peer instead
of from an upstream cache.  This is useful in cluster computing, where
host-to-host bandwidth is high but bandwidth in and out of the cluster is
comparatively low.  It may be cheaper to fetch a chunk from a cluster peer than
an upstream CDN node.

This strategy is also useful for MapReduce~\cite{mapreduce}-style
workflows, where the job scheduler can query gateways to determine
which chunks are already cached so it can schedule jobs on hosts that already
have the requisite data.  This is a feature implemented in Syndicate's HDFS driver, for
example.

This is not part of the default behavior because it makes assumptions about the
network bandwidth between gateways and assumptions about the threat model the
deployment faces.  A wide-area Syndicate volume would not want this feature,
because it would disclose to the Internet information about which gateways could
access which data, and thus give an attacker insight into which hosts
to compromise in order to exfiltrate it.

\hfill \break
\noindent{\textbf{Chunk Compression}}
\hfill \break

Syndicate gives developers the ability to control the wire-format of each chunk.
If the entropy of the data is low, then developers stand to gain by having their
gateways' \texttt{serialize()} and \texttt{deserialize()} driver methods
compress and decompress chunks.  However, if the data has high entropy, then
this strategy would be useless.  Syndicate does not do this by default, but
instead defers to developers to make the right decision based on their data.

\hfill \break
\noindent{\textbf{Read-ahead}}
\hfill \break

Many scientific workflows read sequentially.  If this is the application's
behavior, then the developer can program the Acquire step to pre-fetch blocks
asynchronously.  This is useful if the application is reading
variable-sized ranges of a file that straddle block boundaries---the last block
fetched in one read will be the first block fetched in the next read, so keeping
it local would save a round-trip.

Syndicate does not perform read-ahead by
default because it cannot assume that data reads are sequential.  In a
random-read workload, read-ahead would be more wasteful than the default
behavior.  However, if  the developers know that their application has a
read-sequential access pattern, then they can add this behavior to the Acquire
stage.

\hfill \break
\noindent{\textbf{Favor Shallow Metadata Hierarchies}}
\hfill \break

Developers can reduce the amount of time spent querying metadata by organizing
their data into shallow directory hierarchies.  This would cut down on the
number of round-trips to the MS to resolve a path.  In addition, developers can
ensure that their directories do not get too big in order to minimize the
cold-cache start-up time for a user gateway to synchronize its metadata logs.

This strategy is used in the scientific data-sharing application.

\section{Mutate Flows}

A mutate flow has three steps:  a Build step which constructs a new manifest
for a record that incorporates the modified blocks, a Push step which replicates
the new manifest and new blocks, and a Publish step which makes the mutation
visible to subsequent access flows.  A SDS system supplies default
implementations of these steps, but they may be overwritten by the aggregation
driver.  This section presents the time and space overheads the default steps in
Gaia and Syndicate impose on top of application writes, and presents a
discussion on how developers can minimize them.

\subsection{Overheads in Gaia}

To handle writes, Gaia's default strategy to process a
mutate flow is to do so entirely on the volume owner's device.  When the volume
owner signs into the application, the device's Gaia node instantiates gateways with the Build,
Push, and Publish stages in order to service application writes for this session.

The Build stage takes the new key/value pair the application is trying to write,
and assembles a new key space shard to replicate.  The Push stage takes the
key's value and replicates it to the volume's cloud storage
serivces.  This may include Pushing them to an upstream Gaia node, which carries
out further processing (but to the node doing the Push, the upstream Gaia node
looks and behaves like another cloud storage service).  The Publish stage takes
the new key space shard and replicates it alongside the Pushed key value.

The default deployment of Gaia implements a couple of optmiizations.
First, the Gaia node optimizes the execution of a mutate flow as a sequence
of subroutine calls.  There is minimal overhead between passing control from a
Build stage to a Push stage, and from a Push stage to a Publish stage.
Second, the Push and Publish stages execute in parallel by default.  This is
because there often no logical dependencies between them that require them
to run sequentially.

The end-to-end default write overheads inclue:

\begin{itemize}
\item \textbf{The time and space overheads of generating the new metadata}.  In
the Build step, the Gaia node will need to hash the key/value pair
chunk and append it to the manifest.  This adds a $O(n)$ time overhead, where
$n$ is the size of the value.  In addition, the Gaia node will need to ensure
that it has a fresh copy of the key shard before it can build a new key
shard (i.e. before the mutate flow executes, another mutate flow may have
executed from another one of the user's devices).
\item \textbf{The time and space overheads in storing the new key shard}.  Each new
key added takes $O(1)$ additional space to the volume's manifest, and $O(1)$
additioanl space to the volume's metadata.
Storing the key shard adds a $O(n)$ time and space overhead for $n$ records in the volume
(since in the worst case, a key shard can have as many records as there are
keys in the volume).  These costs are incurred in the Publish step, where the
volume's manifest is replicated.
\end{itemize}

\subsubsection{Measured Overheads}

Write overheads in Gaia were measured on the live Gaia network.  Just as with
the read experiment, this test was conducted on a representative Gaia deployment
whereby the user runs a local Gaia node that will Push and Publish new data to
an upstream Gaia node, which in turn Pushes the data (as chunks) to a bucket in
Microsoft Azure.

The test wrote key/value pairs with sizes ranging between 64K and 640K, in
intervals of 64K, to simulate writing
data from real-world Gaia applications.  The test wrote the files 150 times
using an instrumented Gaia node to measure overheads.  Each run was from a cold
start---there was no caching performed between requests.

\begin{figure}[htp!]
   \centering
   \includegraphics[width=\textwidth]{figures/results/gaia-writes.all/Total}
   \caption{Box-and-whiskers plot of the overall write performance.}
   \label{fig:gaia-write-total}
\end{figure}

The overall write performance for Gaia is shown in
Figure~\ref{fig:gaia-write-total}.  While the measurement is noisy, the write
times increase linearly with the file size.  The source of the noise comes from
the fact that the upstream Gaia ndoe is shared with many Gaia users.

\begin{figure}[htp!]
   \centering
   \begin{subfigure}[b]{.8\textwidth}
      \includegraphics[width=\textwidth]{figures/results/gaia-writes.all/Build}
      \label{fig:gaia-write-build}
      \caption{Gaia Build performance.  Note that this includes the cost of
      fetching the existing manifest before constructing a new one.}
   \end{subfigure}
   \begin{subfigure}[b]{.8\textwidth}
      \includegraphics[width=\textwidth]{figures/results/gaia-writes.all/PushPublish}
      \label{fig:gaia-write-pushpublish}
      \caption{Gaia Push/Publish performance (both stages execute in parallel).}
   \end{subfigure}
   \caption{Box-and-whiskers plots of mutate flow stage performances in Gaia,
   for file sizes between 64K and 640K in increments of 64K.}
   \label{fig:gaia-write-stages}
\end{figure}

Figure~\ref{fig:gaia-write-stages} shows the Build, Push, and Publish stage
perforamnces in Gaia.  In this test, the Build stage includes
the time taken to fetch a copy of the device manifest to update.  If the device
manifest is cached, then the Build stage is extremely fast---effectively the amount of
time taken to hash the data and insert it into a hash table and serialize the
hash table to a string for upload.

The Push and Publish stages run in parallel in Gaia by default, so their
measurements are combined.

\begin{figure}[htp!]
   \centering
   \begin{subfigure}[b]{.8\textwidth}
      \includegraphics[width=\textwidth]{figures/results/gaia-writes.all/Efficiency}
      \label{fig:gaia-read-getmanifest}
      \caption{The cold efficiency, which includes the time taken to fetch the
      volume record over the network.}
   \end{subfigure}
   \begin{subfigure}[b]{.8\textwidth}
      \includegraphics[width=\textwidth]{figures/results/gaia-writes.all/Efficiency_cache}
      \label{fig:gaia-read-discover}
      \caption{The warm efficiency, which assumes the volume record is cached
      and excludes it from the mutate flow's running time}
   \end{subfigure}
   \caption{Box-and-wiskers plots of Gaia's write efficiencies.}
   \label{fig:gaia-write-efficiencies}.
\end{figure}

This test calculated the efficiency of Gaia's write path in two ways---a ``cold
write efficiency'' which includes the cost of fetching the manifest in the Build
stage, and a ``warm write efficiency'' which excludes this step.  Including both
efficiency measures is valuable to developers because often times, the manifest
can be safely cached across writes.  Figure~\ref{fig:gaia-write-efficiencies}
reports both cold and warm efficiencies.  The efficiency of the write path
improves somewhat when the manifest can be cached across writes.

\subsubsection{Recommendations to Developers}

Developers have a few strategies available to alter the performance of writes in Gaia.
The specific strategies taken ultimately depend on the workload and data being
stored.

\hfill \break
\noindent{\textbf{Incremental Key Space Shard Writes}}
\hfill \break

Some applications may have a large key space, but only need to carry out a
key/value writes at a time.
The aggregation driver has an opportunity to reduce the amount of time and space that
need to be consumed to carry the write out by only writing the new key metadata.

If the application only wrote one value, then only one key in the manifest would
be altered.  The Build stage could be optimized to inspect the Gaia node's 
key space shard inbetween writes, only pass along the delta between writes
to the Push stage.

The Push stage would accumulate deltas from the Build stage, and combine them
into a single key shard in the backend storage service.  Then, a subsequent
Acquire step would continue to fetch the key space shard as expected.

The reason this is not the default behavior is because patching a record
efficiently requires the cloud service to support a ``range write''
API call, whereby the client specifies a byte offset and length as to where to
write the given data.  Most popular cloud storage providers do not support
this---they only allow clients to write whole records.  For these services,
a Push stage could not efficiently write key shard deltas, since it would need
to load the entire key shard, patch it, and store the entire updated key shard
on each write.  As such, this behavior would be added by developers in the
special case where they were using a suitable cloud storage provider.

\hfill \break
\noindent{\textbf{Write Batching}}
\hfill \break

Applications may not need all of their writes to be Published immediately.
Instead, a Publish can reflect many writes at once.  This would be allowed if
the application's storage semantics do not require all peers to see each others'
most-recent state.  This can lead to better overall performance for applications
that frequently overwrite the same key/value pairs---overwritten key/value pairs
would not need to be replicated.

Applications that have semantics compatible with write-batching can not only
realize better performance than the default behavior, but also take advantage of
client-side libraries that offer more expressive storage interfaces.
Examples include Compass~\cite{blockstack-compass}, which provides a
MongoDB-like interface, and \texttt{sql.js}~\cite{sql.js}, which provides a
client-side SQLite implementation.  Both of these libraries are easily used with
Gaia, provided that the application's storage semantics allow write-batching
(i.e. a Publish would take place in response to the application committing a
transaction in one of these APIs).  In fact, Compass was designed specifically
for Gaia by a third party contributor.

\subsection{Overheads in Syndicate}

Syndicate's default write strategy is make data as durable as possible. 
This is realized by the default behaviors of replicating all
manifests and blocks to all replica gateways in the volume in the
Push stage, and synchronously uploading the record's metadata to the Syndicate
MS in the Publish stage.

User gateways invoke the Build, Push, and Publish stages on write.
Since Syndicate is designed to process scientific workloads, it expects
multiple write-capable user gateways to be online at once.  However, it assumes
that user gateways usually (but not always) write to the same files that they
coordinate.  This is reasonable in practice, since scientific computing loads
are usually designed to run on many parallel computers which share as little
data with one another as possible.

In light of this, the default common-case behaviors of the Build, Push, and
Publish stages in Syndicate are to assemble a new manifest locally (Build),
replicate the manifest and blocks to all replica gateways (Push), and
synchronously upload the new metadata to the MS (Publish).  Push and Publish are
run automatically when a record is \texttt{close()}'ed, if the application does
not do so explicitly via a call to \texttt{fsync()}.  These are the default
behaviors of the user gateway carrying out the write is also the coordinator.

If the writer gateway is not the coordinator, then it enlists the
coordinator's help it carry out the write.  The writer's Push stage will first
replicate the new blocks, and then synchronously request that the coordinator
both Push a new manifest with the requested changes
and Publish new metadata that reflects it.

By default, replica gateways do not have any specialized aggregation driver
logic on the write path.
They simply accept chunks from user gateways, and replicate them with
their service drivers.

To summarize, the write overheads in Syndicate are as follows:

\begin{itemize}
\item \textbf{The time and space overheads of building a new manifest}.  By
default, a UG will encur a network round-trip when it Builds a new manifest for a record that
it does not coordinate.  In addition, the UG will incure a round-trip to
the MS to ensure that the new manifest it is modifying is fresh when executing
the default Build implementation.
\item \textbf{The time and space overheads of storing new metadata}.  The
record's coordinator will incur a network round-trip to the MS when Publishing
new data, and storing the new metadata incurs $O(1)$ extra space.  In the case
where the writer is not the coordinator, two network round-trips are incurred:
one to the MS and back, and one to the coordinator and back.
\item \textbf{The time and space overheads of storing a new manifest}.  The
record's coordinator will incur a network round-trip to each replica gateway to
store the new manifest, and a network round-trip from each replica gateway to
its underlying storage services.  This yields $O(g)$ round-trips, where $g$ is the number of
replica gateways.  The manifest size is $O(n)$ bytes for a record of $n$ bytes,
so replicating and storing it to all gateways takes $O(gn)$ time and space.
\item \textbf{The time overheads of storing blocks}.  Similar to manifests,
replicating a block takes two network round-trips (one for the replica gateway,
and one for the service).
\end{itemize}

\subsubsection{Measured Overheads}

The overheads of writing in Syndicate were measured for the same block sizes and
record sizes as reads:  records composed of 10 to 100 blocks (in intervals of 10
blocks) for a ``small'' block size of 1K and a ``medium'' block size of 10K.
The same UG, RG, and MS in the read test were used in the write test.
The total mutate flow performances are shown in
Figure~\ref{fig:syndicate-writes-total}.

\begin{figure}[htp!]
   \centering
   \begin{subfigure}[b]{.8\textwidth}
      \includegraphics[width=\textwidth]{figures/results/syndicate-writes.1024.all/Total}
      \label{fig:syndicate-read-discover-1k}
      \caption{Syndicate mutate flow performance with 1K blocks}
   \end{subfigure}
   \begin{subfigure}[b]{.8\textwidth}
      \includegraphics[width=\textwidth]{figures/results/syndicate-writes.1024.all/Total}
      \label{fig:syndicate-read-acquire-1k}
      \caption{Syndicate mutate flow performance with 10K blocks}
   \end{subfigure}
   \caption{Box-and-whiskers plots of mutate flow performances in
   Syndicate, for file sizes between 10K and 100K and a block size of 1K.}
   \label{fig:syndicate-writes-total}
\end{figure}

Despite the noisy measurements, the amount of time taken to write records of
these sizes grows linearly with file size.  For the 1K block measurement, the noise in the
measurements is mainly due to variations in the disk write performance and
chunk-writing scheduler in the UG.  For the 10K block measurement, the noise is
mainly due to the Push stage (i.e. there is more variance in uploading large
blocks).  This is visible in the Build, Push, and Publish
performances in Figures~\ref{fig:syndicate-build}, \ref{fig:syndicate-push}, and
\ref{fig:syndicate-publish}, respectively.

\begin{figure}[htp!]
   \centering
   \begin{subfigure}[b]{.8\textwidth}
      \includegraphics[width=\textwidth]{figures/results/syndicate-writes.1024.all/Build}
      \label{fig:syndicate-build-1k}
      \caption{Syndicate Build stage performance with 1K blocks}
   \end{subfigure}
   \begin{subfigure}[b]{.8\textwidth}
      \includegraphics[width=\textwidth]{figures/results/syndicate-writes.10240.all/Build}
      \label{fig:syndicate-build-10k}
      \caption{Syndicate Build stage performance with 10K blocks}
   \end{subfigure}
   \caption{Box-and-whiskers plots of the Build stage performance, for 1K and
   10K blocks.}
   \label{fig:syndicate-build}
\end{figure}

The Build step occurs within the UG.  In Syndicate, the Build step includes the
process of hashing the blocks and flushing them to a temporary storage location on
disk before it is replicated.  While the effect is hard to see here due to the 
small amount of data, the Build stage's time increases linearly with the amount
of data being written, since the manifest includes the hashes of all blocks
(Figure~\ref{fig:syndicate-build}).  The Build stage with 1K blocks compless in
less than 500 milliseconds, while the Build stage with 10K blocks takes less
than 750 milliseconds.

\begin{figure}[htp!]
   \centering
   \begin{subfigure}[b]{.8\textwidth}
      \includegraphics[width=\textwidth]{figures/results/syndicate-writes.1024.all/Push}
      \label{fig:syndicate-push-1k}
      \caption{Syndicate Push stage performance with 1K blocks}
   \end{subfigure}
   \begin{subfigure}[b]{.8\textwidth}
      \includegraphics[width=\textwidth]{figures/results/syndicate-writes.10240.all/Push}
      \label{fig:syndicate-push-10k}
      \caption{Syndicate Push stage performance with 10K blocks}
   \end{subfigure}
   \caption{Box-and-whiskers plots of the Push stage performance, for 1K and
   10K blocks.}
   \label{fig:syndicate-push}
\end{figure}

The Push stage replicates all blocks to the RG.  The Push stage times show linear
increases with the number of blocks (Figure~\ref{fig:syndicate-push}).
In the 1K block size case, the median Push stage
completes within 1.9 and 2.5 seconds.  In the 10K block size case, the median
Push stage completes within 3.5 and 8.1 seconds.

\begin{figure}[htp!]
   \centering
   \begin{subfigure}[b]{.8\textwidth}
      \includegraphics[width=\textwidth]{figures/results/syndicate-writes.1024.all/Publish}
      \label{fig:syndicate-publish-1k}
      \caption{Syndicate Publish stage performance with 1K blocks}
   \end{subfigure}
   \begin{subfigure}[b]{.8\textwidth}
      \includegraphics[width=\textwidth]{figures/results/syndicate-writes.10240.all/Publish}
      \label{fig:syndicate-publish-10k}
      \caption{Syndicate Publish stage performance with 10K blocks}
   \end{subfigure}
   \caption{Box-and-whiskers plots of the Publish stage performance, for 1K and
   10K blocks.}
   \label{fig:syndicate-publish}
\end{figure}

The Publish stage replicates the new manifest ID to the MS, as well as a
garbage-collection log.  Because the garbage-collection log data that the UG
replicates is proportional to the number of blocks written, it is expected that
the amount of time taken to replicate metadata will increase linearly with the
size of the write.

However, due to the facts that only at most 100 blocks are replicated and that
the size of a garbage-collection entry is 8 bytes, this linear relationship is
not visible.  Publish steps take between 640 milliseconds and 900 milliseconds
across both the 1K and 10K block size tests
(Figure~\ref{fig:syndicate-publish}).

\begin{figure}[htp!]
   \centering
   \begin{subfigure}[b]{.8\textwidth}
      \includegraphics[width=\textwidth]{figures/results/syndicate-reads.1024.all/Efficiency}
      \label{fig:syndicate-efficiency-1k}
      \caption{Syndicate write efficiency for 1K blocks.}
   \end{subfigure}
   \begin{subfigure}[b]{.8\textwidth}
      \includegraphics[width=\textwidth]{figures/results/syndicate-reads.10240.all/Efficiency}
      \label{fig:syndicate-efficiency-1k}
      \caption{Syndicate write efficiency for 10K blocks.}
   \end{subfigure}
   \caption{Box-and-whiskers plots of the efficiencies of Syndicate
   writes.}
   \label{fig:syndicate-write-efficiency}
\end{figure}

The write efficiencies of Syndicate are shown in
Figure~\ref{fig:syndicate-write-efficiency}.  As long as internal fragmentation
can be avoided, using larger block sizes drastically improves the system's
efficiency.

\subsubsection{Recommendations for Developers}

In addition to recommendations for aggregation driver developers for reads, some
performance enhancements can be devised for writes.  These strategies depend on
the workload and the nature of the data, which is why they are not included in
the default behavior.

\hfill \break
\noindent{\textbf{Write Coalescing}}
\hfill \break

A lot of workflows write data sequentially, and in bursts.
Developers can save a set of round-trips to the replica gateways
in the case where two sequential writes straddle a block boundary
by deferring replication of the straddled block.

\hfill \break
\noindent{\textbf{Replica Gateway Selection}}
\hfill \break

Developers are not required to replicate a chunk to all gateways.  It is
expected that in situations where there are multiple choices for a data store,
the aggregation driver will choose which chunk goes with which storage provider.
This can be done both to preserve end-to-end storage semantics, and to improve
write performance.

\hfill \break
\noindent{\textbf{Replica Gateway Chains}}
\hfill \break

Syndicate supports custom gateway types.  Developers can exploit this
to implement chain replication~\cite{chain-replication}~\cite{craq}, whereby a set of 
replica gateways are arranged into a sequence such that when receiving a chunk,
the gateway stores it and forwards it to the next gateway in the sequence.
User gateways would only need to forward chunks to the chain tip.  The tip would
have a ``replica gateway'' type, but the gateways in the chain would have a
distinct ``chain replicator'' type.

The aggregation driver would be written
such that each replica gateway and chain replicator gateway discover their
types and locations in the topology from the certificate graph.  The Push stage
for each gateway would use this knowledge to determine its next-hop gateway.
This strategy is generalizable to arbitrary store-and-forward topologies.

The performance advantage this would incur is that it would enable the same
degree of durability as replicating in parallel, but without the extra
round-trips from the user gateway.  User gateways located behind
underprovisioned network links would notice the improvement, since they would
not need to spend as much time pushing chunks through a local bottleneck.

\hfill \break
\noindent{\textbf{Chunk Patching}}
\hfill \break

If the workload exhibits random-write behaviors, one strategy developers can
emoploy is to implement a ``patching'' algorithm in their aggregation driver's
Push stage.  Instead of sending the entire chunk to a replica gateway, a user
gateway would send only the byte ranges and offsets to the replica gateways.
This would cut down on the data the replica gateway needs to send, even if the
block size in the volume was large.
The replica gateway would reassemble the patches into a block sometime before
the next read occurs---either eagerly as part of an internal
garbage-collection algorithm, or lazily as part of the \texttt{get\_chunk()} or
\texttt{serialize()} driver methods.

\section{Discussion}

Both Syndicate and Gaia add measurable overheads when compared to reading and
writing directly to cloud storage.  This should come as no surprise given the
designs of these two systems.

The overheads in both Gaia and Syndicate are due to three design factors:
all data is broken up and transmitted as blocks and manifests, all data
passes through one or more gateways en route to services that host it, and reads
and writes may incur an extra round-trip to the SDS system's metadata service.
The microbenchmarks presented here show that these overheads either increase the
time and space requirements by a constant factor, or are in a linear
relationship with the amount of data being read or written.  The fact that
the read and write efficiencies both increase with the size of the data indicates
that loading and storing the data to the underlying storage services are the
limiting factors to the system's performance.

The benefits to users and developers the systems offer cannot be overstated.
Using SDS systems has the same value proposition of using TCP/IP sockets instead of
layer-2 frames, or using filesystems instead of directly loading and storing
disk sectors.  While both SDS and these systems impose measureable overhead
and are less performant than the alternatives, the gains that users and
developers realize by using them outweight the performance loss.

The case for SDS-powered applications is becoming more and more apparent to even
non-academic and non-technical audiences.  Implementing featuers such as
end-to-end data privacy and data portability is straightforward in SDS, since
SDS systems already isolate applications from both the storage they use and the
trust relationships between users and organizations.  In fact, Gaia-powered SDS
applications like Graphite Docs and Stealthy have already been reported on in
mainstream media~\cite{graphite-wired}~\cite{washington-post-blockstack}, in
which these very features are touted as technical remedies to problems that
exist in Google Docs and Facebook Messanger, respectively.

\chapter{Related Work}
\label{chap:related-work}

Porting an application to a new service is an inefficient process today,
since the code required to do so is hard to reuse to port another application.
This is because each application and each service have their own \emph{storage
semantics}.  In the limit, each application has its own consistency
model, its own durability model, its own security model, and its own access
methods (i.e. which operations are allowed on a datum, and what side-effects
must occur as a result of its execution).  However, each service
offers its own storage semantics independent of application needs.  Making an application
compatible with a service means writing a patch that translates the
application's semantics into the service's semantics, and vice versa.

The main challenge with porting an application is to ensure that the
its semantics hold end-to-end.  It is not enough to give applications
have a uniform storage access interface, which is the task that today's client-side
portability libraries like Apache libcloud~\cite{libcloud} and on-premesis
storage gateways~\cite{} achieve today.

Making multi-organization applications portable across many storage systems is
important to their survival because each ``piece'' of the application within an
organization must adhere to organization-specific storage requirements.  These
include things like preferred storage providers, user storage quotas, and user access controls.
Their multi-stakeholder nature forces developers to anticipate changing organizational requirements
in the application design, which creates a strong need for cross-system
portability.

Despite this need, combining multiple existing services while preserving end-to-end
storage semantics is non-trivial and if not done without careful consideration.
This can lead to application-specific, non-reusable solutions.  We explore a few examples below:

\noindent{\bf Reading coherent data from a CDN}.  A common way to accelerate
data delivery to readers is to direct read-requests to a CDN.  The CDN
fetches and caches data from upstream application servers, and serves readers
with a replica in order to alleviate load on origin storage servers.

However, introducing CDN in the data path changes the end-to-end data
consistency.  CDNs serve cached data for up to a
server-given or client-requested amount of time.  While serving a cached
replica, the data may change on the backend storage servers, leading to
conflicting views.

Adding a CDN to an application forces developers to either cope with the
weaker consistency model, or devise a way to preserve stronger consistency
in the face of stale data.  Either way, they will need to write an
application-specific patch if they cannot tolerate degrated consistency.
This is because any protocol that can detect and mask stale data requires
coordination between origin servers and clients.  The precise semantics offerred
by this protocol will define the end-to-end consistency model, and will thus be
specific to the application's needs.

\noindent{\bf Reads and Writes to Multiple Storage Providers}.  Developers stand to
improve their data's availability and durability by replicating 
writes to multiple cloud storage services.  However, the replication
logic must take extra constraints into account to be compatible with the
application.  These constraints are application-specific; they include
consistency, replica placement, retention, access logging (and other side-effects),
and domain-specific access controls.

In scientific storage, a wide variety of data is kept in both to local
legacy storage systems and to emerging ``science clouds''.
Even if two workflows use the same services for hosting data, their differring access patterns
can require workflow-specific patches to make them compatible.
For example, and HPC workflow that never generates conflicting writes (i.e. each
worker process writes to its own directory in a shared filesystem) could get away
with replicating data in an eventually-consistent manner.  However, this would
not work for a workflow that required a node to re-use data in previous writes
as part of its computation.

Because scientific workflows interact with a wide variety of potentially
sensitive data, each workflow must implement workflow-specific patches to use
unmodified storage systems.  For example, a workflow that interacts with gene
sequences for deadly influenza strains must take care to log accesses, keep data
encrypted in-flight and at-rest, and store replicas and outputs in highly-secure
networks.  As another example, workflows that interact with patient medical
records must either leverage only HIPPA-compliant storage services (or the legal
equivalent), or implement compliance themselves.  As a third example, workflows
that span multiple wide-area networks (like grid computing or cross-site
sharing) must take care to stage their data in a way that both minimizes access
latency and respects the workflow's consistency (such as Cern
VM-FS~\cite{cern-vm-fs}).  Since addressing each of these concerns requires
knowledge of both the nature of the data being accessed as well as the access
patterns, patches that make workflows compatible with the underlying services
are inevitably specific to the workflow.

\noindent{\bf Continuous Access to Exteranl Datasets}.  Instead of hosting all
of their requisite data themselves, many applications make use of 
externally-curated data sets in a read-only fashion.  Having this data available
reduces the cost of developing each application, since the data only needs to be
gathered once.

However, application developers may find themselves having to 
build out bespoke data access logic to match the dataset service's capabilities
to the application's needs.  For example, an application that makes use
of an underprovisioned data source would implement its own caching layer to
ensure that its reads were low-latency and consistent with the upstream server.
As another example, an application that pays the upstream service provider for
access would need to prevent 3rd parties from accessing it while it holds a copy
(e.g. to deter data piracy).


\chapter{Conclusion}
\label{chap:conclusion}

Wide-area applications that leverage commodity infrastructure are difficult to
keep running in the face of changes in the underlying services.  Services can go
offline, services can change their APIs and storage semantics, and services can
fall outside the trust domains of their users.

We address these problems with wide-area software-defined storage.  By giving
developers the ability to implement their storage semantics as a first-class
storage element, we allow applications to tolerate changes in the underlying
services without requiring a patch each time.  In addition, we reduce the
problem of keeping many applications compatible with a single services to making a
service compatible with the software-defined storage system, instead of patching
each application.

We present the design space of wide-area software-defined storage, and
distilled several design principles for building such systems.  We showed the
feasibility of designing real-world SDS systems by creating two
implementations---Gaia and Syndicate.  Both systems allow applications to
leverage commodity cloud services in the face of changes to both the service API
and changes to the trust relationships users have with them.

To demonstrate the feasibility of constructing SDS-powered applications, we
present the design and implementation of three real-world applications:
end-to-end encrypted Webmail, decentralized groupware, and CDN-accelerated
scientific data staging.  In all three applications, we show how the ability to
define an aggregation driver allows us to solve several hard problems that have
plagued prior systems.

We give microbenchmarks and early performance numbers for our SDS prototypes and
sample applications.  We show that the overhead of the SDS system is acceptable,
since it does not affect the sample applications' usability.
Gaia, Syndicate, and our sample applications have all been released as
open-source~\cite{blockstack-core}~\cite{syndicate-storage}~\cite{syndicatemail}~\cite{todo-list}~\cite{blockstack-browser}~\cite{syndicate-sdm}~\cite{syndicate-containers}.




% Make the bibliography single spaced
\singlespacing
\bibliographystyle{plain}

% add the Bibliography to the Table of Contents
\cleardoublepage
\ifdefined\phantomsection
  \phantomsection  % makes hyperref recognize this section properly for pdf link
\else
\fi
\addcontentsline{toc}{chapter}{Bibliography}

\bibliography{bibdata}

\end{document}
