\chapter{Conclusion}
\label{chap:conclusion}

Wide-area applications that leverage commodity infrastructure are difficult to
keep running in the face of changes in the underlying services.  Services can go
offline, services can change their APIs and storage semantics, and services can
fall outside the trust domains of their users.

We address these problems with wide-area software-defined storage.  By giving
developers the ability to implement their storage semantics as a first-class
storage element, we allow applications to tolerate changes in the underlying
services without requiring a patch each time.  In addition, we reduce the
problem of keeping many applications compatible with a single services to making a
service compatible with the software-defined storage system, instead of patching
each application.

We present the design space of wide-area software-defined storage, and
distilled several design principles for building such systems.  We showed the
feasibility of designing real-world SDS systems by creating two
implementations---Gaia and Syndicate.  Both systems allow applications to
leverage commodity cloud services in the face of changes to both the service API
and changes to the trust relationships users have with them.

To demonstrate the feasibility of constructing SDS-powered applications, we
present the design and implementation of three real-world applications:
end-to-end encrypted Webmail, decentralized groupware, and CDN-accelerated
scientific data staging.  In all three applications, we show how the ability to
define an aggregation driver allows us to solve several hard problems that have
plagued prior systems.

We give microbenchmarks and early performance numbers for our SDS prototypes and
sample applications.  We show that the overhead of the SDS system is acceptable,
since it does not affect the sample applications' usability.
Gaia, Syndicate, and our sample applications have all been released as
open-source~\cite{blockstack-core}~\cite{syndicate-storage}~\cite{syndicatemail}~\cite{todo-list}~\cite{blockstack-browser}~\cite{syndicate-sdm}~\cite{syndicate-containers}.

