\chapter{Design Principles of Software-define Storage}
\label{chap:design_principles}

Software-defined storage (SDS) is currently a buzzword with loose meaning.  We
define it as a programmable collection of compute nodes that process
\texttt{get}, \texttt{put}, and \texttt{delete} requests on bound-size chunks of
data.

In terms of programmatic expressiveness, this storage layer is strictly more capable
than a cloud service client library (like Apache libcloud~\cite{libcloud)), but
less capable than a general-purpose VM (Figure~\ref{fig:intro-sds-complexity}).  Unlike a general-purpose
VM, this layer offers ``just enough'' of a programming interface to efficiently define,
deploy, run, and re-use logic to enforce the user's ownership and control how other
users interact with the data.  Unlike a cloud service library, the storage volumes can take actions \emph{proactively}
and \emph{asynchronously} to curate the user data, such as migrating it between
providers or generating usage reports.  By building storage systems
that fall into this middle complexity category, we spare the user from having to
administrate full-fledged servers while giving them direct programmatic control
over how other users may interact with their data.  We call storage systems that
follow this design principle \emph{wide-area software-defined storage systems}.

Wide-area SDS systems are architecturally similar to wide-area software-defined networks
(SDNs).  The user's trusted devices run a set of programmable ``storage
switches'' called \emph{gateways} that work together to create virtual storage
volumes backed by untrusted commodity storage infrastructure.  The gateways are
the ingress and egress points for reads and writes on the volume's data---like
network function virtualization in SDNs, they validate, process, and transform
read and write requests.  In addition, they run storage logic that
may take independent actions to curate
the user's data, if the user so desires.  Users program their gateways using an
cloud service similar to an SDN controller called a \emph{Metadata Service}, which
they use to fetch, authenticate, and apply user-signed storage logic to implement
the user's overall storage processing rules.

